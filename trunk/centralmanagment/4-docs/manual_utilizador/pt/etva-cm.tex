\chapter{\textsf{ETVA}}
\section{Descrição}
A \emph{Eurotux Virtual Appliance} é uma ferramenta de gestão centralizada de recursos disponíveis numa rede. Consiste numa distribuição linux pré-instalada e configurada que permite fazer a gestão via rede de servidores e seus recursos.


A ETVA encontra-se dividida principalmente em dois blocos funcionais:

\begin{itemize}
	\item \emph{Central Management} (CM)
        \item \emph{Virtualization Agent} (VA)
\end{itemize}

\begin{figure}[H]
	\begin{center}
	\includegraphics[scale=0.35]{screenshots/etva_blocos.png}
	\caption{Esquema geral do ETVA}
	\label{fig:etva_blocos}
	\end{center}
\end{figure}

O CM é o bloco responsável por gerir toda a infra-estrutura.
Os \emph{Virtualization Agents} são responsáveis pelo processamento dos pedidos entre os servidores de virtualização (\emph{nodes}) e o CM.

Dentro de um servidor de virtualização(\emph{node}) poderão existir máquinas virtuais com \emph{Management Agents}. Estes agentes, permitem a gestão ao nível dos serviços/aplicações instalados numa máquina virtual (ver figura \ref{fig:etva_blocos} ).

\section{Versões}

Atualmente o ETVA encontra-se disponível em duas versões:
\begin{description}
	\item[Standard -] Nesta versão, o modelo do ETVA consiste num único servidor de virtualização onde se encontram instalados o CM e o VA. A configuração da rede do \emph{node} neste modelo consistem em quatro interfaces de rede: Internet, LAN, DMZ e Management.
		\begin{figure}[H]
			\begin{center}
			\includegraphics[scale=0.4]{screenshots/etva_standard.png}
			\caption{Modelo ETVA Standard}
			\label{fig:etva_standard}
			\end{center}
		\end{figure}
	\item[Enterprise -] Nesta versão, existem vários servidores de virtualização (\emph{nodes}) a comunicar com o CM. A configuração da rede inicial, é efectuada, com recurso a VLANs, através do \emph{Assistente de configuração inicial} conforme indica a figura \ref{fig:first_time_wizard}.
		\begin{figure}[H]
			\begin{center}
			\includegraphics[scale=0.6]{screenshots/etva_enterprise.png}
			\caption{Modelo ETVA Enterprise}
			\label{fig:etva_enterprise}
			\end{center}
		\end{figure}
\end{description}
 
Este manual de utilização/configuração descreve a ferramenta de gestão do ETVA, o CM (\emph{Central Management}).

\pagebreak
\chapter{\textsf{Instalação}}
\label{chp:installation}
\section{Versão standard}

Para efectuar a instalação deveremos ligar a appliance à electricidade, um teclado, um monitor e uma drive CD USB externa.
Depois deverá ser iniciado o boot a partir do cd e teremos a seguinte imagem:

\begin{figure}[H]
	\begin{center}
	\includegraphics[scale=0.6]{screenshots/install_etva1.png}
	\caption{Menu de instalação da versão ETVA Standard}
	\label{fig:boot_install_screen_standard}
	\end{center}
\end{figure}

De seguida, seleccionando a opção \emph{"Install ETVA-SMB-KVM (This will erase all disks)"} iniciar-se-á o arranque da instalação conforme:

\begin{figure}[H]
	\begin{center}
	\includegraphics[scale=0.4]{screenshots/install_etva2.png}
    \includegraphics[scale=0.4]{screenshots/install_etva3.png}    
    \includegraphics[scale=0.4]{screenshots/install_etva4.png}
    \includegraphics[scale=0.4]{screenshots/install_etva5.png}    
    \includegraphics[scale=0.4]{screenshots/install_etva6.png}        
    \includegraphics[scale=0.4]{screenshots/install_etva8.png}
    \includegraphics[scale=0.4]{screenshots/install_etva9.png}
    \includegraphics[scale=0.4]{screenshots/install_etva10.png}
\caption{Instalando a versão ETVA Standard}
	\label{fig:installation_standard}
	\end{center}
\end{figure}

Depois da instalação efectuada o arranque deverá ser efectuado a partir do disco rigido e deverá aparecer uma imagem como a seguinte:

\begin{figure}[H]
	\begin{center}
	\includegraphics[scale=0.5]{screenshots/install_etva11.png}
	\caption{Menu de boot da versão ETVA Standard}
	\label{fig:boot_screen_standard}
	\end{center}
\end{figure}

No final da instalação deve-se ligar um cabo de rede do nosso PC de acesso à porta \emph{Management} conforme a figura \ref{fig:back_standard}.

\begin{figure}[H]
	\begin{center}
	\includegraphics[scale=0.12]{screenshots/appliance_back_identifica.jpg}
	\caption{Identificação das portas a ligar na versão ETVA Standard}
	\label{fig:back_standard}
	\end{center}
\end{figure}

De seguida configura-se a placa de rede do nosso PC:

\begin{quote}
Endereço IP: 10.10.4.1\\
Máscara: 255.255.255.0
\end{quote}

Finalmente, abre-se o nosso browser e acede-se ao seguinte endereço:
\begin{quote}
http://10.10.4.254/
\end{quote}

\pagebreak
\chapter{\textsf{ETVA Central Management}}

\section{Estrutura da interface principal}
O layout principal é constituido por quatro áreas:

\begin{description}
	\item[Painel topo -] Possui menus de acesso a acções do sistema, tais como a administração de utilizadores, gestão de ISOs e visualização das mensagens do sistema.
	\item[Painel esquerdo (\emph{Nodes}) -] Lista as máquinas reais/servidores de virtualização - {\bf\emph{nodes}} e as máquinas virtuais associadas a cada \emph{node} - {\bf\emph{servidores}}. No nível imediatamente abaixo de \emph{Main} encontram-se os vários servidores de virtualização registados no CM. As funcionalidades permitidas num servidor de virtualização estão descritas na secção \ref{sec:node}. No nível abaixo de um \emph{node} encontram-se as máquinas virtuais do respectivo \emph{node}. As funcionalidades de uma máquina virtual encontram-se descritas na secção \ref{sec:server}. Ao clicar em cada item é carregada a informação correspondente no painel principal.
	\item[Painel principal -] Área onde é visualizada o conteúdo pretendido, consoante o contexto (item a visualizar).
	\item[Painel de informação (\emph{Painel de Informação}) -] Área de breve notificação acerca dos eventos despoletados pelo utilizador. Mensagens de erro e sucesso são aqui visualizadas.
\end{description}

\begin{figure}[H]
	\begin{center}
	\includegraphics[scale=0.45]{screenshots/principal.png}
	\caption{Layout principal}
	\label{fig:principal}
	\end{center}
\end{figure}

\pagebreak


\section{Primeiro acesso}
\label{sec:first_access}
Após a instalação do CM pela primeira vez acede-se ao url do sistema disponível no endereço http://<ENDEREÇO IP>\footnote{Endereço especificado na {\bf \emph{Instalação}} (capítulo \ref{chp:installation}).}

\begin{figure}[H]
	\begin{center}
	\includegraphics[scale=0.7]{screenshots/login.png}
	\caption{Página de autenticação}
	\label{fig:login}
	\end{center}
\end{figure}

A página de autenticação é disponibilizada e deverá ser introduzido o \emph{Username} e a respectiva \emph{Password}. Também é possível seleccionar o idioma do sistema\footnote{De momento apenas estão disponíveis os idiomas Português e Inglês}.

\begin{quote}
	{\large \bf Nota} \\*[-.8pc]
	\underline{\hspace{6in}} \\
	Ao instalar o CM pela primeira vez as credenciais de acesso são:
	\begin{description}
        	\item[Username:] admin
	        \item[Password:] admin
	\end{description}
	Por questões de segurança recomenda-se a alteração da password do sistema no primeiro acesso através do assistente de configuração inicial.

\end{quote}

No primeiro acesso ao \emph{Central Management} deverá surgir o \emph{Assistente de configuração inicial} que permite efectuar a configuração incial do sistema (ver secção \ref{sec:first_time_wizard}).

De seguida, e após a instalação e configuração de um agente virtualização num \emph{node}, este regista-se automáticamente no CM, passando o CM a dispor de mais funcionalidades.
No painel esquerdo, \emph{Nodes} (ver figura \ref{fig:principal}), surgirá o servidor de virtualização registado no CM e poderá então passar-se a efectuar a gestão desse \emph{node} conforme as opções descritas na secção \ref{sec:node}.

\pagebreak

\section{Main}

Neste painel é apresentada a vista geral do CM.
Podemos visualizar os servidores de virtualização e a informação da rede do CM (ver figura \ref{fig:main_nodes}).

\subsection{Nodes}

Em \emph{Nodes} é disponibilizada alguma informação acerca dos vários servidores de virtualização. Podemos ver o \emph{hypervisor} suportado pelas máquinas reais e, entre outras informações, o estado do agente de virtualização.
\begin{figure}[H]
	\begin{center}
	\includegraphics[scale=0.45]{screenshots/main_nodes.png}
	\caption{Vista dos nodes do Central Management}
	\label{fig:main_nodes}
	\end{center}
\end{figure}

\subsection{Redes}

Este painel permite efectuar as seguintes operações sobre o CM:

\begin{itemize}
	\item Administração das redes do sistema
	\item Gestão da pool de endereços MAC
	\item Gestão das interfaces de rede das máquinas virtuais 
\end{itemize}

\begin{figure}[H]
	\begin{center}
	\includegraphics[scale=0.45]{screenshots/main_networks.png}
	\caption{Vista das redes do sistema  e das interfaces de rede}
	\label{fig:main_networks}
	\end{center}
\end{figure}

É possível também filtrar as interface de rede numa determinada rede clicando sobre a rede pretendida conforme a figura \ref{fig:main_networks}.
Na figura \ref{fig:main_networks} as interfaces de rede listadas são as que estão associadas à rede \emph{Internet}

\subsubsection{Administração das redes}

Para criar uma rede clica-se em \emph{Adicionar rede}.
A informação da rede consiste no seu nome e ID\footnote{Caso a rede/vlan seja \emph{tagged} o campo \emph{ID da rede} refere-se à \emph{VLAN ID}} (ver figura \ref{fig:network_create}).

Para remover uma rede selecciona-se a rede pretendida e clica-se em \emph{Remover rede}.

\begin{quote}
	{\large \bf Nota} \\*[-.8pc]
	\underline{\hspace{6in}} \\
	As operações de adicionar/remover rede só estão disponíveis na versão \emph{ETVA Enterprise}.
\end{quote}


\begin{figure}[H]
	\begin{center}
	\includegraphics[scale=0.5]{screenshots/network_create.png}
	\caption{Janela de criação de uma rede}
	\label{fig:network_create}
	\end{center}
\end{figure}

A rede adicionada/removida é propagada a todos os \emph{nodes} do CM.


\subsubsection{Gestão da pool de endereços MAC}
\label{sec:mac_pool}

Em \emph{Gestão da Pool de MAC} (ver figura \ref{fig:main_networks}), é possivel criar a pool de endereços MAC.
Para além de adicionar MACs à pool, pode-se visualizar as redes associadas e os MACs ainda disponíveis da pool.

\begin{figure}[H]
	\begin{center}
	\includegraphics[scale=0.5]{screenshots/networks_macpool.png}
	\caption{Janela de criação da pool de MACs}
	\label{fig:networks_macpool}
	\end{center}
\end{figure}


\subsubsection{Gestão das interfaces de rede das máquinas virtuais}
Seleccionando um registo da tabela de interfaces e acedendo ao sub-menu de contexto, é possível remover a interface de rede associada a esse registo - \emph{Remover interface de rede}, ou alterar as interfaces de rede da máquina virtual associada ao registo seleccionado - \emph{Gestão das interfaces de rede}.

\begin{figure}[H]
	\begin{center}
	\includegraphics[scale=0.5]{screenshots/nics.png}
	\caption{Janela de gestão das interfaces de rede de uma máquina virtual}
	\label{fig:nics}
	\end{center}
\end{figure}

Na gestão de interfaces de uma máquina, dependendo do tipo de máquina virtual é possível seleccionar os drivers das placas de rede.\footnote{Esta opção está disponível para máquinas em HVM ou KVM.Os drivers disponíveis são: e1000, rtl8139 e virtio}

% PAINEL NODE

\section{Servidor de virtualização}
\label{sec:node}

No painel \emph{Nodes} é possivel seleccionar um \emph{node}(servidor de virtualização), e efectuar as seguintes operações:
\begin{itemize}
    \item Visualizar informação do \emph{node} (ver secção \ref{sec:nodeinfo})
    \item Gestão de máquinas virtuais (ver secção \ref{sec:servers})
    \item Gestão do armazenamento do node (ver secção \ref{sec:storage})
\end{itemize}

Para além das operações mencionadas acima, é possível aceder ao sub-menu de contexto de um \emph{node} que permite operações de:
\begin{itemize}
    \item Carregar node
    \item Opções de conectividade\footnote{Disponível apenas na versão \emph{ETVA Enterprise}}
    \item Alterar keymap
    \item Estado do node
\end{itemize}

Em \emph{Opções de conectividade}, é possível editar a configuração da interface \emph{Management} ao qual se encontra ligado o agente de virtualização.
\begin{figure}[H]
	\begin{center}
	\includegraphics[scale=0.5]{screenshots/node_conn.png}
	\caption{Configuração da conectividade do agente \emph{VirtAgent01}}
	\label{fig:node_conn}
	\end{center}
\end{figure}

Em \emph{Alterar keymap}, consoante o item seleccionado, servidor de virtualização ou máquina virtual, é possível definir o keymap padrão usado pelo VNC, ou o keymap específico a uma determinada máquina virtual respectivamente.

Em \emph{Estado do node}, é possível enviar ao servidor de virtualização um pedido de verificação do estado da conectividade do agente.

\subsection{Informação do node}
\label{sec:nodeinfo}
Em \emph{Informação do node} é disponibilizada a informação acerca do servidor de virtualização. Podemos ver o \emph{hypervisor} suportado pela máquina real e, entre outras informações, o estado do agente de virtualização.

\begin{figure}[H]
	\begin{center}
	\includegraphics[scale=0.45]{screenshots/node_info.png}
	\caption{Informação do node \emph{VirtAgent01}}
	\label{fig:node_info}
	\end{center}
\end{figure}

\subsection{Servidores}
\label{sec:servers}
Em \emph{Servidores} é disponibilizada a informação acerca das máquinas virtuais existente no servidor de virtualização. Para além de visualizar informação, este painel permite efectuar as seguintes operações:
\begin{itemize}
	\item Adicionar máquina virtual
    \item Editar máquina virtual
	\item Remover maquina virtual
	\item Abrir máquina virtual numa consola VNC
	\item Iniciar/parar máquina virtual
    \item Migrar máquina virtual
\end{itemize}
\begin{figure}[H]
	\begin{center}
	\includegraphics[scale=0.45]{screenshots/node_servers.png}
	\caption{Lista das máquinas virtuais do node \emph{VirtAgent01}}
	\label{fig:node_servers}
	\end{center}
\end{figure}

\subsubsection{Adicionar máquina virtual}
\label{sec:add_server}

Para adicionar uma nova máquina virtual utiliza-se o botão \emph{Assistente de criação de servidor}.
\begin{quote}
	{\large \bf Nota} \\*[-.8pc]
	\underline{\hspace{6in}} \\
	As opções deste painel só se encontram activas se o agente de virtualização estiver a correr no \emph{node} (máquina real) e este conseguir estabelecer comunicação com o CM.
\end{quote}
 

\begin{figure}[H]
	\begin{center}
	\includegraphics[scale=0.5]{screenshots/server_createwiz.png}
	\caption{Assistente de criação de servidor - Bem-vindo}
	\label{fig:server_createwiz}
	\end{center}
\end{figure}
Este assistente é constituido pelas seguintes etapas:
\begin{description}
	\item[Nome da máquina virtual:] Nesta etapa define-se o nome da máquina virtual e o tipo de sistema operativo. As opções do sistema operativo variam consoante a especificação do node:
		\begin{itemize}
			\item com XEN e suporte a virtualização por hardware:
			\begin{itemize}
				\item Linux PV
				\item Linux HVM
				\item Windows
			\end{itemize}
 			\item com XEN sem suporte de virtualização por hardware:
			\begin{itemize}
				\item Linux PV
			\end{itemize}
 			\item com KVM
			\begin{itemize}
				\item Linux
				\item Windows
			\end{itemize}
		\end{itemize}
	
		\begin{figure}[H]
        		\begin{center}
		        \includegraphics[scale=0.5]{screenshots/server_createwiz_name.png}
        		\caption{Assistente de criação de servidor - Nome da máquina virtual}
	        	\label{fig:server_createwiz_name}
	        	\end{center}
		\end{figure}
 
	\item[Memória:] Especificação da memória a ser usada pela máquina.
		\begin{figure}[H]
        		\begin{center}
		        \includegraphics[scale=0.5]{screenshots/server_createwiz_memory.png}
        		\caption{Assistente de criação de servidor - Memória}
	        	\label{fig:server_createwiz_memory}
	        	\end{center}
		\end{figure}

	\item[Processador:] Nesta etapa define-se o número de processadores a usar.
		\begin{figure}[H]
        		\begin{center}
		        \includegraphics[scale=0.5]{screenshots/server_createwiz_processor.png}
        		\caption{Assistente de criação de servidor - Processador}
		        \label{fig:server_createwiz_processor}
	        	\end{center}
		\end{figure}

	\item[Armazenamento:] Define o disco de arranque da máquina virtual. Pode ser uma das três opções:
\begin{itemize}
	\item usar um logical volume/ficheiro já existente - \emph{Logical volume existente}
	\item criar um novo logical volume/ficheiro (para criar um ficheiro através desta opção tem que se seleccionar o volume group \emph{\_\_DISK\_\_}\footnote{Ver secção \ref{sec:storage}}) - \emph{Novo logical volume}
	\item  ou caso pretenda criar um ficheiro usar a opção \emph{Novo ficheiro} que para tal necessita apenas do nome e tamanho.
\end{itemize}

        \begin{figure}[H]
        		\begin{center}
	        	\includegraphics[scale=0.5]{screenshots/server_createwiz_storage.png}
	        	\caption{Assistente de criação de servidor - Armazenamento}
		        \label{fig:server_createwiz_storage}
        		\end{center}
		\end{figure}

		\begin{quote}
			{\large \bf Nota} \\*[-.8pc]
			\underline{\hspace{6in}} \\
			Se o \emph{node} não suportar \emph{physical volumes} a opção \emph{Logical volume existente} será desabilitada, uma vez que não é possivel criar \emph{logical volumes}, mas sim apenas ficheiros.
		\end{quote}		
        
        
        \item[Rede do servidor:] Especificação das interfaces de rede existentes no servidor. Caso não existam endereços MAC disponíveis é possível criar através de \emph{Gestão da Pool de MAC}. Igualmente para as redes é possível criar nesta etapa através de \emph{Adicionar rede}.
		\begin{figure}[H]
        		\begin{center}
	        	\includegraphics[scale=0.5]{screenshots/server_createwiz_hostnet.png}
	        	\caption{Assistente de criação de servidor - Rede do servidor}
		        \label{fig:server_createwiz_hostnet}
        		\end{center}
		\end{figure}

        \item[Arranque:] Especificação de parâmetros de arranque da máquina virtual. As opções nesta etapa variam consoante o tipo de sistema definido na etapa \emph{Nome da máquina virtual}:
		\label{sec:add_server_boot}
        \begin{itemize}
			\item \emph{Linux PV}
				\begin{itemize}
					\item Instalação via rede. Url do kernel a carregar.
				\end{itemize}
			\item Outros
				\begin{itemize}
					\item Boot de rede (PXE)
					\item CD-ROM (ISO)
				\end{itemize}
		\end{itemize}
        A figura \ref{fig:server_createwiz_startup} refere-se às opções de uma máquina virtual em \emph{Linux PV}.

		\begin{figure}[H]
			\begin{center}
			\includegraphics[scale=0.5]{screenshots/server_createwiz_startup.png}
			\caption{Assistente de criação de servidor - Arranque}
			\label{fig:server_createwiz_startup}
			\end{center}
		\end{figure}

	\item[Finalização!] Etapa final do assistente. Após confirmação da criação do servidor, os dados recolhidos nas etapas anteriores são processados e enviados ao servidor de virtualização. Posteriormente no painel \emph{Servidores} poderá ser iniciada a máquina através da opção \emph{Iniciar servidor}.
		\begin{figure}[H]
			\begin{center}
			\includegraphics[scale=0.5]{screenshots/server_createwiz_finish.png}
			\caption{Assistente de criação de servidor - Finalização!}
			\label{fig:server_createwiz_finish}
			\end{center}
		\end{figure}

\end{description}

\subsubsection{Editar máquina virtual}
\label{sec:edit_server}
Para editar um servidor, selecciona-se a máquina pretendida e clica-se em \emph{Editar servidor}.

    \begin{quote}
        {\large \bf Nota} \\*[-.8pc]
        \underline{\hspace{6in}} \\
        Se a máquina virtual for do tipo PV, é possivel fazer alterações com a máquina a correr, caso contrário a opção encontra-se desabilitada, sendo necessário que a máquina não se encontre activa para poder efectuar alterações.
    \end{quote}

A edição de uma máquina virtual permite a configuração de:
\begin{description}
	\item[Opções gerais:] Neste painel é permitido alterar o nome, memória, opções do keymap e parâmetros de boot da máquina.
        Os parâmetros de boot variam consoante o tipo da máquina virtual (ver secção \ref{sec:add_server_boot}).
		\begin{figure}[H]
        		\begin{center}
		        \includegraphics[scale=0.5]{screenshots/server_edit_general.png}
        		\caption{Edição de um servidor - Opções gerais}
	        	\label{fig:server_edit_general}
	        	\end{center}
		\end{figure}

	\item[Interfaces de rede:] Adicionar/remover interfaces. É possível alterar o tipo de driver a usar se aplicável\footnote{Só é possível especificar o driver a usar se a máquina virtual for HVM ou KVM}.
		\begin{figure}[H]
        		\begin{center}
		        \includegraphics[scale=0.5]{screenshots/server_edit_interfaces.png}
        		\caption{Edição de um servidor - Interfaces de rede}
	        	\label{fig:server_edit_interfaces}
	        	\end{center}
		\end{figure}

	\item[Discos:] Adicionar/remover discos da máquina. A máquina virtual tem que ter pelo menos um disco associado.
                    Para adicionar/remover discos selecciona-se o disco pretendido e recorre-se ao \emph{drag-n-drop} entre as tabelas.
                    
                \begin{quote}
                    {\large \bf Nota} \\*[-.8pc]
                    \underline{\hspace{6in}} \\
                    O disco de boot da máquina é o disco que se encontra na primeira posição da tabela.
                \end{quote}
                    
		\begin{figure}[H]
        		\begin{center}
		        \includegraphics[scale=0.5]{screenshots/server_edit_disks.png}
        		\caption{Edição de um servidor - Discos}
		        \label{fig:server_edit_disks}
	        	\end{center}
		\end{figure}

\end{description}



\subsubsection{Remover máquina virtual}
\label{sec:remove_server}
Para remover um servidor, selecciona-se a máquina a remover e clica-se em \emph{Remover servidor}.

A opção \emph{Manter disco} permite manter o disco associado à máquina aquando da sua criação, caso contrário será também removido.
		
\begin{figure}[H]
	\begin{center}
	\includegraphics[scale=0.5]{screenshots/server_remove.png}
	\caption{Janela de remoção de um servidor}
	\label{fig:server_remove}
	\end{center}
\end{figure}

\subsubsection{Abrir máquina virtual numa consola VNC}
\label{sec:open_vnc}

Seleccionando um servidor e de seguida clicando em \emph{Abrir numa consola} é possível estabelecer uma ligação VNC com a máquina, desde que esta esteja a correr.

\begin{quote}
	{\large \bf Nota} \\*[-.8pc]
	\underline{\hspace{6in}} \\
	Caso o teclado esteja desconfigurado é possível alterar o \emph{keymap} do VNC através da opção \emph{Alterar keymap} no sub-menu de contexto do painel \emph{Nodes}.
    O \emph{keymap} pode ser definido quer ao nível de cada servidor, ou definir um \emph{keymap} de uso geral, o qual será usado por omissão na criação de novas máquinas virtuais.
\end{quote}

\subsubsection{Iniciar/parar máquina virtual}
\label{sec:start_server}

No arranque da máquina virtual é possível escolher um dos seguintes parâmetro de boot:
\begin{description}
	\item[Disco:] Arranque pelo disco associado ao servidor.
    \item[PXE:] Arranque por PXE\footnote{Só disponível caso o tipo da máquina virtual não seja \emph{Linux PV}\label{foot:notpv}}.
    \item[Location URL:] Arranque pelo url definido em Location\footnote{Só disponível caso o tipo da máquina virtual seja \emph{Linux PV}}.
	\item[CD-ROM:] Arranque pela imagem montada no CD-ROM\footref{foot:notpv}.
    	 
\end{description}

\begin{figure}[H]
	\begin{center}
	\includegraphics[scale=0.45]{screenshots/server_start.png}
	\caption{Parâmetros de arranque de uma máquina virtual}
	\label{fig:server_start}
	\end{center}
\end{figure}


\subsubsection{Migrar máquina virtual}
\label{sec:migrate_server}

Seleccionando um servidor e de seguida clicando em \emph{Migrar servidor} é possível migrar uma máquina de um \emph{node} para outro desde que partilhem o mesmo armazenamento.
A migração de uma máquina virtual é efectuada no modo \emph{offline}.

\begin{figure}[H]
	\begin{center}
	\includegraphics[scale=0.5]{screenshots/server_migrate.png}
	\caption{Migração de uma máquina virtual}
	\label{fig:server_migrate}
	\end{center}
\end{figure}

\begin{quote}
	{\large \bf Nota} \\*[-.8pc]
	\underline{\hspace{6in}} \\
	Esta opção só está disponível no modelo \emph{ETVA Enterprise}.
\end{quote}


\subsection{Armazenamento}
\label{sec:storage}

Em \emph{Armazenamento} encontra-se a informação relativa aos volumes existentes no \emph{node}.
Este painel encontra-se divido em três secções:

\begin{description}
	\item[Devices -] Informação relativa aos \emph{physical volumes}\footnote{Um \emph{physical volume} é um dispositivo fisico, como por exemplo um disco} e seu estado. Permite fazer a administração de \emph{physical volumes} do \emph{node}.
	\item[Volume Groups -] Lista os \emph{volumes groups}\footnote{Um \emph{volume group} consiste na agregação de diversos \emph{physical volumes} num único volume virtual} existentes no node e seus \emph{physical volumes} associados. Permite fazer operações de administração de \emph{volume groups}.
	\item[Logical Volumes -] Apresenta a informação dos \emph{logical volumes}\footnote{Um \emph{logical volume} é uma "fatia" de um \emph{volume group}. É usado como sendo uma partição do sistema} do \emph{node}. Área de administração dos \emph{logical volumes}.
\end{description}


\begin{quote}
	{\large \bf Nota} \\*[-.8pc]
	\underline{\hspace{6in}} \\
	Existe um \emph{volume group} especial, \_\_DISK\_\_, utilizado no manuseamento de ficheiros. Esta etiqueta serve para, aquando da criação de um \emph{logical volume}, indicar que o disco a ser usado não é de facto um \emph{logical volume} mas sim um ficheiro.
\end{quote}


\begin{figure}[H]
	\begin{center}
	\includegraphics[scale=0.45]{screenshots/node_storage.png}
	\caption{Informação do armazenamento de um \emph{node}}
	\label{fig:inicial}
	\end{center}
\end{figure}

% ADMINISTRAÇÃO PHYSICAL VOLUMES

\subsubsection{Administração de Physical Volumes}
A administração de \emph{physical volumes} consiste nas seguintes operações:
\begin{itemize}
	\item Inicialização de um \emph{physical volume}
        \item Remoção da inicialização de um \emph{physical volume}
\end{itemize}

\begin{figure}[H]
        \begin{center}
        \includegraphics[scale=0.45]{screenshots/node_storage_device_ctx.png}
        \caption{Sub-menu de contexto de um physical volume}
        \label{fig:storage_device_ctx}
        \end{center}
\end{figure}


Para inicializar um \emph{physical volume} acede-se ao sub-menu de contexto do \emph{device} pretendido e seleccionar \emph{Inicializar physical volume}. Para remover um \emph{physical volume} a operação é análoga, bastando seleccionar a opção \emph{Remover inicialização do physical volume} no sub-menu de contexto do \emph{physical volume}.

\begin{quote}
	{\large \bf Nota} \\*[-.8pc]
	\underline{\hspace{6in}} \\
	Só é permitido remover um \emph{physical volume} se este não pertencer a nenhum \emph{volume group}.
\end{quote}

% ADMINISTRAÇÃO VOLUME GROUPS

\subsubsection{Administração de Volume Groups}
Na administração de \emph{volumes groups} é permitido:
\begin{itemize}
	\item Criar um \emph{volume group}
	\item Extender um \emph{volume group}
	\item Reduzir um \emph{volume group}
	\item Remover um \emph{volume group}
\end{itemize}

\begin{figure}[H]
        \begin{center}
        \includegraphics[scale=0.45]{screenshots/node_storage_vg_ctx.png}
        \caption{Sub-menu de contexto de um volume group}
        \label{fig:storage_vg_ctx}
        \end{center}
\end{figure}

Para criar um \emph{volume group} acede-se ao sub-menu de contexto sobre um qualquer \emph{volume group} e seleccionar \emph{Adicionar volume group}.
Na janela de criação deverá ser introduzido o nome pretendido e seleccionar um ou mais \emph{physical voumes} disponíveis.

Um \emph{physical volume} está disponível quando não está alocado a nenhum \emph{volume group} e encontra-se inicializado.

\begin{figure}[H]
        \begin{center}
        \includegraphics[scale=0.5]{screenshots/storage_vg_create.png}
        \caption{Janela de criação de um volume group}
        \label{fig:storage_vg_create}
        \end{center}
\end{figure}

Para extender um \emph{volume group} recorre-se ao \emph{drag-n-drop}, ou seja, arrasta-se o \emph{physical volume}, que se pretende adicionar, para cima do \emph{volume group} pretendido.

Na remoção/redução de um \emph{volume group} seleccciona-se o \emph{volume group}/\emph{physical volume} a remover e escolhe-se a opção correspondente do sub-menu de contexto.
\begin{quote}
	{\large \bf Nota} \\*[-.8pc]
	\underline{\hspace{6in}} \\
	Só é permitido remover um \emph{volume group} se não houver nenhum \emph{logical volume} associado ao \emph{volume group}.
\end{quote}
 
\begin{figure}[H]
        \begin{center}
        \includegraphics[scale=0.45]{screenshots/storage_vg_extend.png}
        \caption{Extensão de um volume group}
        \label{fig:storage_vg_extend}
        \end{center}
\end{figure}

Na figura \ref{fig:storage_vg_extend} extende-se o \emph{volume group} {\bf black} com o \emph{physival volume} {\bf sdb1}.

% ADMINISTRAÇÃO LOGICAL VOLUMES

\subsubsection{Administração de Logical Volumes}

As operações disponíveis sobre os \emph{logical volumes} são as seguintes:
\begin{itemize}
	\item Criar um \emph{logical volume}
	\item Redimensionar um \emph{logical volume}
	\item Remover um \emph{logical volume}
\end{itemize}

\begin{figure}[H]
        \begin{center}
        \includegraphics[scale=0.45]{screenshots/node_storage_lv_ctx.png}
        \caption{Sub-menu de contexto de um logical volume}
        \label{fig:storage_lv_ctx}
        \end{center}
\end{figure}

Para criar um \emph{logical volume} acede-se ao sub-menu de contexto sobre um qualquer \emph{logical volume} e selecciona-se \emph{Adicionar logical volume}.
Na janela de criação deverá ser introduzido o nome pretendido, o \emph{volume group} a partir do qual se criará e o tamanho que não deverá exceder o tamanho disponível no \emph{volume group}.

\begin{figure}[H]
        \begin{center}
        \includegraphics[scale=0.5]{screenshots/storage_lv_create.png}
        \caption{Janela de criação de um logical volume}
        \label{fig:storage_lv_create}
        \end{center}
\end{figure}

No redimensionamento selecciona-se o \emph{logical volume} que se pretende redimensionar e acede-se ao sub-menu de contexto. Aí existe a opção \emph{Redimensionar logical volume} que permite aumentar/reduzir o tamanho do \emph{logical volume}.


\begin{quote}
	{\large \bf Nota} \\*[-.8pc]
	\underline{\hspace{6in}} \\
	Ao reduzir o tamanho do \emph{logical volume} poderá tornar os dados existentes inutilizados. É da responsabilidade do utilizador verificar se é comportável/seguro o redimensionamento do \emph{logical volume} sem afectar os dados nele contidos.
\end{quote}


\begin{figure}[H]
        \begin{center}
        \includegraphics[scale=0.5]{screenshots/storage_lv_resize.png}
        \caption{Redimensionamento de um logical volume}
        \label{fig:storage_lv_resize}
        \end{center}
\end{figure}

Na remoção de um \emph{logical volume}, no sub-menu de contexto existe a opção \emph{Remover logical volume}. O \emph{logical volume} só será removido se não tiver associado a nenhuma máquina virtual. Para verificar se está em uso passa-se o rato por cima do \emph{logical volume} e observar a informação contida no \emph{tooltip} que aparece.

\pagebreak

\section{Máquina virtual}
\label{sec:server}

No painel \emph{Nodes} é possível seleccionar a máquina virtual sobre o qual pretendemos efectuar operações como:

\begin{itemize}
        \item Gestão da máquina virtual
        \item Visualizar estatísticas        
        \item Gestão dos serviços do \emph{Management Agent}
\end{itemize}

\subsection{Informação do servidor}
Em \emph{Informação do servidor} podemos ver o estado da máquina virtual e, entre outras informações, o estado do \emph{Management Agent}.
Para além de visualizar informação, este painel permite efectuar as seguintes operações:
\begin{itemize}
	\item Adicionar máquina virtual (ver secção \ref{sec:add_server})
    \item Editar máquina virtual (ver secção \ref{sec:edit_server})
	\item Remover máquina virtual (ver secção \ref{sec:remove_server})
	\item Abrir máquina virtual numa consola VNC (ver secção \ref{sec:open_vnc})
	\item Iniciar/parar máquina virtual (ver secção \ref{sec:start_server})
    \item Migrar máquina virtual (ver secção \ref{sec:migrate_server})
\end{itemize}

\begin{figure}[H]
	\begin{center}
	\includegraphics[scale=0.45]{screenshots/server_info.png}
	\caption{Informação da máquina virtual \emph{etfww}}
	\label{fig:server_info}
	\end{center}
\end{figure}

\subsection{Estatísticas}
Em \emph{Estatísticas} é possível visualizar gráficamente informação de:
\begin{itemize}
	\item Cpu Usage
	\item Networks
	\item Memory Usage
	\item Disk
	\item Node Load
\end{itemize}


\begin{figure}[H]
	\begin{center}
	\includegraphics[scale=0.45]{screenshots/server_stats_nodeLoad.png}
	\caption{Estatísticas de uma máquina virtual}
	\label{fig:server_stats_nodeLoad}
	\end{center}
\end{figure}

Em cada um destes paineis é possível visualizar os dados pelos intervalos pré-definidos:
\begin{itemize}
	\item Última hora
	\item Últimas 2 horas
	\item Últimas 24 horas
	\item Última semana
\end{itemize}

Na figura \ref{fig:server_stats_nodeLoad}, visualiza-se a informação de carga do node a que pertence o servidor \emph{etfww} para o intervalo - \emph{Última hora}.

Para visualizar outros intervalos de tempo usa-se \emph{Gerar imagem do gráfico}. A imagem gerada é conforme a figura \ref{fig:server_stats_nodeLoadRange}.
\begin{figure}[H]
	\begin{center}
	\includegraphics[scale=0.5]{screenshots/server_stats_nodeLoadRange.png}
	\caption{Estatísticas de \emph{Utilização do node} - Carga no CPU}
	\label{fig:server_stats_nodeLoadRange}
	\end{center}
\end{figure}

\subsection{Serviços}
Em \emph{Serviços}, e caso esteja configurado um MA (\emph{Management Agent}) no servidor, é disponibilizada a respectiva configuração dos serviços controlados por esse MA.

\section{Ferramentas}

No menu \emph{Ferramentas} é possível aceder às seguintes ferramentas:
\begin{itemize}
\item Importar OVF
\item Exportar OVF
\item Gestor de ISOs
\item Monitorização do agente dos nodes
\item Registo de eventos do sistema
\end{itemize}

\subsection{Importar OVF}
Esta ferramenta permite importar máquinas virtuais no formato OVF (\emph{Open Virtualization Format}).

\begin{figure}[H]
	\begin{center}
	\includegraphics[scale=0.5]{screenshots/ovf_import.png}
	\caption{Assistente de importação OVF - Bem-vindo}
	\label{fig:ovf_import_wiz}
	\end{center}
\end{figure}

O assistente de importação OVF é constituido pelas seguintes etapas:

\begin{description}
	\item[Ficheiro OVF de origem:] Nesta etapa define-se o URL do ficheiro OVF a importar (ver figura \ref{fig:ovf_import_file}).
		
        \begin{quote}
            {\large \bf Nota} \\*[-.8pc]
            \underline{\hspace{6in}} \\
            O CM tem que ter acesso via HTTP ao URL especificado.
        \end{quote}

        \begin{figure}[H]
            \begin{center}
            \includegraphics[scale=0.5]{screenshots/ovf_import_file.png}
            \caption{Assistente de importação OVF - Ficheiro OVF de origem}
            \label{fig:ovf_import_file}
            \end{center}
        \end{figure}

	\item[Resumo do OVF:] Detalhes do ficheiro OVF. Disponibiliza informação acerca do produto, versão, tamanho total dos ficheiros referenciados pelo OVF, se disponível.
		\begin{figure}[H]
            \begin{center}
            \includegraphics[scale=0.5]{screenshots/ovf_import_resume.png}
            \caption{Assistente de importação OVF - Resumo do OVF}
            \label{fig:ovf_import_resume}
            \end{center}
        \end{figure}

    \item[Contrato de licença:] Se especificado no ficheiro OVF, esta etapa surgirá com o EULA. Caso contrário, esta etapa será omitida.
		\begin{figure}[H]
            \begin{center}
            \includegraphics[scale=0.5]{screenshots/ovf_import_eula.png}
            \caption{Assistente de importação OVF - Contrato de licença}
            \label{fig:ovf_import_eula}
            \end{center}
        \end{figure}

    \item[Nome e localização:] Nesta etapa define-se o nome da máquina virtual, o node de destino e o tipo de sistema operativo. As opções do sistema operativo variam consoante a especificação do node:
		\begin{itemize}
			\item com XEN e suporte a virtualização por hardware:
			\begin{itemize}
				\item Linux PV
				\item Linux HVM
				\item Windows
			\end{itemize}
 			\item com XEN sem suporte de virtualização por hardware:
			\begin{itemize}
				\item Linux PV
			\end{itemize}
 			\item com KVM
			\begin{itemize}
				\item Linux
				\item Windows
			\end{itemize}
		\end{itemize}

        \begin{figure}[H]
            \begin{center}
            \includegraphics[scale=0.5]{screenshots/ovf_import_name.png}
            \caption{Assistente de importação OVF - Nome e localização}
            \label{fig:ovf_import_name}
            \end{center}
        \end{figure}

        Antes de prosseguir para a próxima etapa, o assistente verifica se os drivers para os discos e para as interfaces de rede mencionados no OVF são suportados pelo servidor de virtualização escolhido.

        Os drivers dos discos suportados para máquinas XEN com ou sem virtualização por hardware são: ide, xen e scsi. Nas máquinas KVM os drivers são: ide, virtio e scsi.

        Os drivers da placa de rede suportados para máquinas em HVM ou KVM são: e1000, rtl8139 e virtio. Numa máquina XEN sem suporte a virtualização nao suporta drivers.

        Caso o servidor de virtualização escolhido não suporte os drivers mencionados no OVF a importação não poderá ser efectuada.


    \item[Armazenamento:] Nesta etapa é efectuado o mapeamento dos discos no node. É possível especificar o nome a dar ao \emph{logical volume} bem como definir o \emph{volume group}.
        É necessário que todo os discos sejam mapeados para prosseguir para a próxima etapa.
		\begin{figure}[H]
            \begin{center}
            \includegraphics[scale=0.5]{screenshots/ovf_import_storage.png}
            \caption{Assistente de importação OVF - Armazenamento}
            \label{fig:ovf_import_storage}
            \end{center}
        \end{figure}    

    \item[Interfaces de rede:] Nesta etapa é efectuado o mapeamento das interfaces de rede. É possível especificar novas interfaces de rede.
        É necessário que todas as interfaces de rede sejam mapeadas para prosseguir para a próxima etapa.
		\begin{figure}[H]
            \begin{center}
            \includegraphics[scale=0.5]{screenshots/ovf_import_networks.png}
            \caption{Assistente de importação OVF - Interfaces de rede}
            \label{fig:ovf_import_networks}
            \end{center}
        \end{figure}

    \item[Finalização!] Etapa final do assistente. Após confirmação da importação da máquina virtual, os dados recolhidos nas etapas anteriores são processados e enviados ao servidor de virtualização. Posteriormente no painel \emph{Servidores} poderá ser iniciada a máquina através da opção \emph{Iniciar servidor}.
		\begin{figure}[H]
			\begin{center}
			\includegraphics[scale=0.5]{screenshots/ovf_import_finish.png}
            \caption{Assistente de importação OVF - Finalização!}
			\label{fig:ovf_import_finish}
			\end{center}
		\end{figure}

\end{description}

\subsection{Exportar OVF}
Esta ferramenta permite exportar máquinas virtuais no formato OVF (\emph{Open Virtualization Format}).
O ficheiro gerado vem no formato OVA (\emph{Open Virtualization Archive}).

\begin{quote}
	{\large \bf Nota} \\*[-.8pc]
	\underline{\hspace{6in}} \\
	A máquina virtual a exportar necessita estar parada para se efectuar a exportação.
\end{quote}

\begin{figure}[H]
	\begin{center}
	\includegraphics[scale=0.5]{screenshots/ovf_export.png}
	\caption{Janela de exportação OVF}
	\label{fig:ovf_export}
	\end{center}
\end{figure}


\subsection{Gestor de ISOs}
Esta ferramenta permite fazer a gestão das imagens que estão disponíveis para uso nas máquinas virtuais.
Os ficheiros existentes servirão posteriormente para serem montadas no \emph{CD-ROM} das máquinas virtuais.

\begin{figure}[H]
	\begin{center}
	\includegraphics[scale=0.5]{screenshots/iso_manager.png}
	\caption{Painel de gestão das ISOs}
	\label{fig:iso_manager}
	\end{center}
\end{figure}

As operações permitidas são:
\begin{itemize}
\item Upload de múltiplos ficheiros
\item Download de ficheiros
\item Renomear ficheiros
\item Apagar ficheiros
\end{itemize}


\begin{quote}
	{\large \bf Nota} \\*[-.8pc]
	\underline{\hspace{6in}} \\
	As alterações efectuadas às imagens existentes, que estejam definidas no arranque por CD-ROM de uma qualquer máquina virtual,
     não se irão reflectir automáticamente. Cabe ao utilizador verificar se a imagem montada no CD-ROM continua válida.
\end{quote}

\subsection{Monitorização do agente dos nodes}
Esta ferramenta serve para verificar em tempo real a comunicação dos vários nodes com o CM. A verificação é feita periódicamente.
Para parar a verificação fecha-se o pop-up que surge aquando da activação da ferramenta.

\subsection{Registo de eventos do sistema}

Em \emph{Registo de eventos do sistema} é possível visualizar as interações efectuadas entre o utilizador, nodes, servidores e o CM.

\begin{figure}[H]
	\begin{center}
	\includegraphics[scale=0.5]{screenshots/events_log.png}
	\caption{Janela do registo de eventos do sistema}
	\label{fig:events_log}
	\end{center}
\end{figure}

As mensagens do registo de eventos podem ser filtradas por três tipos de mensagem:
\begin{itemize}
    \item {\bf Debug} - Apresenta todas as mensagens. Agrega os níveis \emph{Info} e \emph{Error}
    \item {\bf Info} - Mensagens com informação dos eventos que foram bem sucedidos
    \item {\bf Error} - Mensagens com informação dos eventos que não foram bem sucedidos
\end{itemize}

\section{Administração do sistema}
\label{sec:first_time_wizard}
No menu \emph{Administração do sistema} é possível efectuar:
\begin{itemize}
\item Assistente de configuração inicial
\item Alterar preferências
\item Administração de utilizadores e permissões
\end{itemize}

\subsection{Assistente de configuração inicial}
O assistente de configuração inicial reúne o conjunto de operações a efectuar no primeiro acesso ao CM. Permite efectuar uma primeira configuração rápida do sistema.

O assistente de configuração, conforme a figura \ref{fig:first_time_wizard}, consiste nos seguintes passos:
\begin{itemize}
	\item Alteração da password inicial
	\item Geração da MAC pool
	\item Configuração da Rede
\end{itemize}

\begin{figure}[H]
        \begin{center}
        \includegraphics[scale=0.7]{screenshots/first_time_wizard.png}
        \caption{Assistente de configuração inicial (\emph{ETVA Enterprise})}
        \label{fig:first_time_wizard}
        \end{center}
\end{figure}

\begin{quote}
	{\large \bf Nota} \\*[-.8pc]
	\underline{\hspace{6in}} \\
	Caso se trate de um \emph{ETVA Standard} a configuração das redes é omitida.
\end{quote}

\subsection{Alterar preferências}
Acedendo a \emph{Preferências} é possível definir alguns parâmetros globais ao sistema.
No painel \emph{Geral} é permitido especificar o \emph{keymap} usado por omissão no acesso por VNC às máquinas virtuais, bem como definir a duração dos registos de eventos do sistema.

\begin{figure}[H]
        \begin{center}
        \includegraphics[scale=0.5]{screenshots/preferences_general.png}
        \caption{Janela de preferências do sistema - Painel Geral}
        \label{fig:preferences_general}
        \end{center}
\end{figure}

No painel \emph{Conectividade} é permitido alterar o IP do CM e da rede LAN caso se trate de um \emph{ETVA Enterprise}. No modelo \emph{ETVA Enterprise} permite apenas configurar o IP do CM.

\begin{figure}[H]
        \begin{center}
        \includegraphics[scale=0.5]{screenshots/preferences_conn.png}
        \caption{Janela de preferências do sistema - Painel Conectividade}
        \label{fig:preferences_conn}
        \end{center}
\end{figure}

\subsection{Admininstração de utilizadores e permissões}
Na Administração de utilizadores e permissões é possivel fazer a gestão de contas de utilizadores, permissões e grupos.

\begin{figure}[H]
        \begin{center}
        \includegraphics[scale=0.4]{screenshots/user_admin.png}
        \caption{Janela de administração de utilizadores e permissões}
        \label{fig:user_admin}
        \end{center}
\end{figure}

O CM actualmente não suporta diferentes tipos de credenciais\footnote{As credenciais serão implementadas numa futura versão} apesar de ser possível criar vários grupos de utilizadores e permissões.
Todos os utilizadores podem efectuar a gestão de utilizadores e permissões.

\begin{quote}
	{\large \bf Nota} \\*[-.8pc]
	\underline{\hspace{6in}} \\    
	Em \emph{Gestão de Grupos} não é possível remover o grupo com ID 1 dado ser o grupo por omissão do sistema.
\end{quote}

\pagebreak
\chapter{\textsf{ETVA Management Agents}}

Com o ETVA é possível gerir via interface web os serviços de algumas soluções Eurotux nomeadamente:
\begin{itemize}
	\item ETFW - Solução de Firewall
	\item ETMS - Solução de Mail Server
	\item ETVOIP - Solução de VOIP
\end{itemize}
    
    
A gestão dos serviços é possível através da instalação e configuração de agentes de gestão (\emph{Management Agents}) nas máquinas virtuais onde correm as respectivas soluções.

%\section{ETFW}
\section{ETFW}
In a virtual machine with the image of \textit{ETFW} you can configure the service via Central Management, on tab \textit{ETFW}.

In the main panel you can access the \textit{Network setup wizard} or access the \textit{Webmin} interface to make changes in any configuration.

The site also has the \textit{Save configuration} option that makes the changes effective. Without this option, any changes made lost after \textit{reboot}.

\begin{figure}[H]
    \begin{center}
    \includegraphics[scale=0.38]{screenshots/etfw/etfwmain.png}
    \caption{Main panel}
    \label{fig:etfwmain}
    \end{center}
\end{figure}

\subsection{Network setup Wizard}

To configure the module \textit{ETFW} quickly and efficiently it's provided a step by step setup wizard.

\begin{figure}[H]
    \begin{center}
    \includegraphics[scale=0.38]{screenshots/etfw/etfw_wizard_01.png}
    \caption{Welcome}
    \label{fig:etfw_wizard_passo1}
    \end{center}
\end{figure}

\begin{figure}[H]
    \begin{center}
    \includegraphics[scale=0.38]{screenshots/etfw/etfw_wizard_02.png}
    \caption{Topology setup}
    \label{fig:etfw_wizard_passo2}
    \end{center}
\end{figure}

\begin{figure}[H]
    \begin{center}
    \includegraphics[scale=0.38]{screenshots/etfw/etfw_wizard_03.png}
    \caption{Configuring the WAN interface}
    \label{fig:etfw_wizard_passo3}
    \end{center}
\end{figure}

\begin{figure}[H]
    \begin{center}
    \includegraphics[scale=0.38]{screenshots/etfw/etfw_wizard_04.png}
    \caption{Configuring the LAN interface}
    \label{fig:etfw_wizard_passo4}
    \end{center}
\end{figure}

\begin{figure}[H]
    \begin{center}
    \includegraphics[scale=0.38]{screenshots/etfw/etfw_wizard_05.png}
    \caption{Configuring of the DHCP service for LAN network}
    \label{fig:etfw_wizard_passo5}
    \end{center}
\end{figure}

\begin{figure}[H]
    \begin{center}
    \includegraphics[scale=0.38]{screenshots/etfw/etfw_wizard_06.png}
    \caption{Configuring SQUID proxy}
    \label{fig:etfw_wizard_passo6}
    \end{center}
\end{figure}

\begin{figure}[H]
    \begin{center}
    \includegraphics[scale=0.38]{screenshots/etfw/etfw_wizard_07.png}
    \caption{Configuring DMZ interface}
    \label{fig:etfw_wizard_passo7}
    \end{center}
\end{figure}

\begin{figure}[H]
    \begin{center}
    \includegraphics[scale=0.38]{screenshots/etfw/etfw_wizard_08.png}
    \caption{Configuring DHCP service for DMZ network}
    \label{fig:etfw_wizard_passo8}
    \end{center}
\end{figure}

\begin{figure}[H]
    \begin{center}
    \includegraphics[scale=0.38]{screenshots/etfw/etfw_wizard_09.png}
    \caption{Completion of the configuration}
    \label{fig:etfw_wizard_passo9}
    \end{center}
\end{figure}

\subsection{Network setup - \textit{Network}}
In addition to the wizard configuration process you can change the network configuration manually.
To do this, go to the \textit{network} tab where we have access to the configuration of network interfaces (\textit{Network interfaces}), routing rules (\textit{Routing and gateways}), address configuration (\textit{Host Addresses}) and client \textit{DNS} (\textit{Hostname and DNS client}).

\subsubsection{Network interfaces}
In \textit{Network interfaces} we can see the network interfaces that are configured and which are going to be active at the machine start.

When selecting an interface we can edit the parameters of the interface, such as the IP address, netmask, broadcast, and aliases in virtual interfaces.
To add a new interface, select the add button and fill in the required fields for the interface configuration.

\begin{figure}[H]
    \begin{center}
    \includegraphics[scale=0.38]{screenshots/etfw/etfw_network_interfaces_01.png}
    \caption{Active interfaces}
    \label{fig:etfw_network_interfaces_01}
    \end{center}
\end{figure}

\begin{figure}[H]
    \begin{center}
    \includegraphics[scale=0.38]{screenshots/etfw/etfw_network_interfaces_02.png}
    \caption{After boot active interfaces}
    \label{fig:etfw_network_interfaces_02}
    \end{center}
\end{figure}

If you wish to define an alias for a network interface, you must select the desired interface, edit and choose \textit{Add virtual} in \textit{Virtual interfaces}. Next, fill out the required parameters and save.

\begin{figure}[H]
    \begin{center}
    \includegraphics[scale=0.38]{screenshots/etfw/etfw_network_interfaces_03.png}
    \caption{Alias interface}
    \label{fig:etfw_network_interfaces_02}
    \end{center}
\end{figure}

\begin{figure}[H]
    \begin{center}
    \includegraphics[scale=0.38]{screenshots/etfw/etfw_network_interfaces_04.png}
    \caption{Alias interface active at boot}
    \label{fig:etfw_network_interfaces_02}
    \end{center}
\end{figure}

\subsubsection{Routing and gateways}
In \textit{Routing and gateways} we can query the active routing rules and remove or add new rules.
We can also define routing rules that are set at startup.

\begin{figure}[H]
    \begin{center}
    \includegraphics[scale=0.38]{screenshots/etfw/etfw_network_routing_01.png}
    \caption{Active forwarding rules}
    \label{fig:etfw_network_routing_01}
    \end{center}
\end{figure}

To create a routing rule, we choose the \textit{add} option and define the parameters of the route:

\begin{itemize}
    \item \textit{Route destination} - Destination of the forwarding route. It can be \textit{default}, destination IP address or network address;
    \item \textit{Netmask for destination} - Netmask for the route of destination;
    \item \textit{Route via} - \textit{Gateway} or network interface of output route.
\end{itemize}

\begin{figure}[H]
    \begin{center}
    \includegraphics[scale=0.38]{screenshots/etfw/etfw_network_routing_02.png}
    \caption{Defined routing rules at startup}
    \label{fig:etfw_network_routing_02}
    \end{center}
\end{figure}

In the routing rules defined at startup we can configure the following rule types:
\begin{itemize}
    \item \textit{Default routes} - to define the default gateway;
    \item \textit{Static routes} - static routes to other networks/machines;
    \item \textit{Local routes} - static routes where IP addresses are defined, netmask and the interface of each route.
\end{itemize}

\subsubsection{Host Addresses}
In \textit{Host Addresses} you can define static names and locals, associated with IP addresses, in order to reduce response time in name resolution.

\begin{figure}[H]
    \begin{center}
    \includegraphics[scale=0.38]{screenshots/etfw/etfw_network_hostaddresses_01.png}
    \caption{Address setup}
    \label{fig:etfw_network_hostaddresses_01}
    \end{center}
\end{figure}

\subsubsection{Hostname and DNS Client}
From this option you can set the name that the machine will have locally and the configuration parameters of the DNS client service (DNS server addresses, order of name resolution on addresses and search domains names).

\begin{figure}[H]
    \begin{center}
    \includegraphics[scale=0.38]{screenshots/etfw/etfw_network_dnsclient_01.png}
    \caption{DNS client}
    \label{fig:etfw_network_dnsclient_01}
    \end{center}
\end{figure}

\subsection{Firewall rules}
By accessing the tab \textit{Firewall} we have the ability to set the rules of Firewall.
This firewall is based on \textit{iptables}, formed by three basic objects:

\begin{itemize}
    \item Rules
    \item Chains
    \item Tables
\end{itemize}

The rules are lower-level objects that perform packet filtering or manipulation.
A rule consists of the following parts:

\begin{itemize}
\item The table were the rule will be added;
\item The chain chain to which the rule will be added;
\item The filtering or manipulation instructions.
\end{itemize}

The rules are organized in chains and act as a checklist of rules ordered.

The chains are organized in tables that group a large number of possible rules to filter and/or manipulate packets.

The operation of the firewall proceeds as follows: if the packet header meet the requirements of the rule, this will follow the destiny imposed by the rule, otherwise it will be evaluated further by the next rule.
%When there are no more rules will be applied the package the default rule or standard set.

\begin{figure}[H]
    \begin{center}
    \includegraphics[scale=0.38]{screenshots/etfw/etfw_firewall_01.png}
    \caption{Firewall: Filter table}
    \label{fig:etfw_firewall_01}
    \end{center}
\end{figure}

By accessing the firewall tab the interface shows three tables: \textit{Packet alteration} - \textit{mangle}, \textit{Network address translation} - \textit{nat} and \textit{Packet filtering} - \textit{filter}.

\subsubsection{Table \textit{Filter} - \textit{Packet Filtering}}
The table \textit{filter} is used to filter packets that pass through the firewall and presents three firewall pre-defined chains:

\begin {itemize}
   \item \textbf{INPUT} - filter packets whose destination is the own firewall;
   \item \textbf{OUTPUT} - filters packets whose origin is firewall;
   \item \textbf{FORWARD} - filter packets that pass through the firewall which are not the source nor the destination of the firewall.
\end {itemize}

\begin{figure}[H]
    \begin{center}
    \includegraphics[scale=0.38]{screenshots/etfw/etfw_firewall_02.png}
    \caption{Creating a rule on table \textit{filter} - Chain and action}
    \label{fig:etfw_firewall_02}
    \end{center}
\end{figure}

When creating a rule in a filter table's chain some parameters are required for action to take:

\begin{itemize}
   \item \textbf{Rule comment} - allows you to write a short comment to identify the rule to be created;
   \item \textbf{Action to take} - action to be taken, who decides what should be done with the package, if it matches the rule. The most important actions are:
       \subitem \textbf{Drop} - removes the packet without doing anything else;
       \subitem \textbf{Accept} - the firewall lets the packet pass through and the data is sent to the recipient;
       \subitem \textbf{Reject} - works the same way as the drop but is sent an \textit {ICMP} error back to the sender of the package;
       \subitem \textbf{Userspace} - if this option is active, a multicast is donne by kernel into a socket where can be a listening process.
\end{itemize}

\begin{figure}[H]
    \begin{center}
    \includegraphics[scale=0.38]{screenshots/etfw/etfw_firewall_03.png}
    \caption{Creating a rule on table \textit{filter} - Condition details}
    \label{fig:etfw_firewall_03}
    \end{center}
\end{figure}

Other options that can be completed in the condition are the following:
\begin{itemize}
    \item \textbf{Source address or network} - Network address that establish the origin of the package. It is usually a combination of IP address with the subnet mask separated by a slash (eg 192.168.1.0/255.255.255.0 or 192.168.1.0/24);
    \item \textbf{Destination address or network} - Network address to which the packet is intended. It has the same combination as before;
    \item \textbf{Incoming interface} - Specifies the input interface of the package;
    \item \textbf{Outgoing interface} - Specifies the output interface of the package;
    \item \textbf{Fragmentation} - Sometimes a packet undergoes fragmentation because of its size, being its part joined together later in the destination. This option sets whether the rule is matched by fragmented packets or not;
    \item \textbf{Network protocol} - Network protocol;
    \item \textbf{Source TCP or UDP port} - Sets the match rule with the incoming port (tcp or udp), and may be defined a range of ports (example: 1000:1050) or a list doors separated by commas;
    \item \textbf{Destination TCP or UDP port} - Sets the output port match and may be offered a range of ports (example: 1000:1050) or a list of ports separated by commas;
    \item \textbf{Source and destination port(s)} - Sets the door match, both incoming and outgoing, may be offered a range of ports or a list of ports separated by commas;
    \item \textbf{ICMP packet type} - ICMP packet type;
    \item \textbf{Ethernet address} - Defines a physical address of a network that will serve to match the rule;
    \item \textbf{Packet flow rate} - Sets the volume of packages that will make the rule match;
    \item \textbf{Packet burst rate} - Sets the threshold at which the packets begin to make match with the rule;
    \item \textbf{Connection states} - Specifies the number of links that make the rule match;
    \item \textbf{Type of service} - Defines the type of service for which we want the rule to do match;
    \item \textbf{Additional parameters} - Specifies additional parameters that will be passed directly to the line of the rule to be applied.
\end{itemize}

\subsubsection{NAT table - Network address translation}
The NAT table is used for network address translation, or translate a package with a particular field of origin or destination.
Only the first packet will be affected by this chain, after which the remaining packages will apply the same actions as the first.
This table presents three pre-defined chains:

\begin{itemize}
    \item \textbf{PREROUTING} - applies the changes to the packages when the target needs to be changed;
    \item \textbf{POSTROUTING} - applies the changes to the packages when the source needs to be changed;
    \item \textbf{OUTPUT} - applies the changes to packets originated by the firewall.
\end{itemize}

The current targets in this table are:

\begin{itemize}
    \item \textbf{DNAT} - used in cases where you have a public IP address in the firewall and if you want to redirect the access to another host (a DMZ for example), thus allowing forward traffic.
    \item \textbf{SNAT} - used when you want to change the origin of package, usually to hide the addresses of local network or \textit{DMZ}.
    \item \textbf{MASQUERADE} - used the same way that the SNAT and for the same reason, but the outgoing IP address is not specified, by using the source address of the interface package. This rule is used primarily for dynamic IP addresses, because if the link goes down, the source address that was being used is discarded yielding place to a new source address of the interface when the connection is restored.
\end{itemize}

\begin{figure}[H]
    \begin{center}
    \includegraphics[scale=0.38]{screenshots/etfw/etfw_firewall_04.png}
    \caption{Creating a rule on NAT table}
    \label{fig:etfw_firewall_03}
    \end{center}
\end{figure}

The options used in the NAT table are identical to the Filter table. The differences are listed bellow:

\begin{itemize}
    \item \textbf{Action to take} - action to be taken, has the same functionality described in the Filter table, but has two different options:
        \subitem \textbf{Masquerade} - Rewrites the outgoing IP address, when it comes to dynamic IP addresses;
        \subitem \textbf{Source NAT} / \textbf{Destination NAT} - Depending on the chain (PREROUTING, POSTROUTING or OUTPUT), available options may vary. They will be \textit{Source NAT} and \textit{estination NAT}, respectively. Also, they will re-write the IP's input and output respectively.
\end{itemize}

\subsubsection{Mangle table - Packet alteration}

This table is not addressed in this version of the manual.

\subsection{DHCP Server}
In this tab we can configure the DHCP server, including IP range to allocate the hosts, router address, DNS server address, view active leases, and to start/stop the server.

\begin{figure}[H]
    \begin{center}
    \includegraphics[scale=0.38]{screenshots/etfw/etfw_dhcp_wizard_01.png}
    \caption{IP range setup}
    \label{fig:etfw_dhcp_wizard_01}
    \end{center}
\end{figure}

From the DHCP Wizard we can, in a first stage, set the IP ranges allocated to hosts.

\begin{figure}[H]
    \begin{center}
    \includegraphics[scale=0.38]{screenshots/etfw/etfw_dhcp_subnets_01.png}
    \caption{Subnets setup}
    \label{fig:etfw_dhcp_subnets_01}
    \end{center}
\end{figure}

\begin{figure}[H]
    \begin{center}
    \includegraphics[scale=0.38]{screenshots/etfw/etfw_dhcp_subnets_02.png}
    \caption{Subnet edit}
    \label{fig:etfw_dhcp_subnets_02}
    \end{center}
\end{figure}

Later, you can edit the subnets and set the appropriate parameters to our setting for the network address, netmask, address range, among others.

\begin{figure}[H]
    \begin{center}
    \includegraphics[scale=0.38]{screenshots/etfw/etfw_dhcp_interfaces_01.png}
    \caption{Choose an interface}
    \label{fig:etfw_dhcp_interfaces_01}
    \end{center}
\end{figure}

To configure the server properly you must set the interface on which service will operate.
Go to \textit{Edit Network Interface} and choose the desired network interface.

\begin{figure}[H]
    \begin{center}
    \includegraphics[scale=0.38]{screenshots/etfw/etfw_dhcp_texteditor_01.png}
    \caption{Configuration file edition}
    \label{fig:etfw_dhcp_texteditor_01}
    \end{center}
\end{figure}

Additionally, you can always view and edit the configuration file directly and put the desired options.

\begin{figure}[H]
    \begin{center}
    \includegraphics[scale=0.38]{screenshots/etfw/etfw_dhcp_leases_01.png}
    \caption{List of active leases}
    \label{fig:etfw_dhcp_leases_01}
    \end{center}
\end{figure}

In any time you can find a list of assigned IPs by selecting the option \textit{List active leases}. This option is also available for each subnet.

\subsection{SQUID server}

In the \textit{SQUID Server} tab you can configure the SQUID Proxy service.

\begin{figure}[H]
    \begin{center}
    \includegraphics[scale=0.38]{screenshots/etfw/etfw_squid_wizard_01.png}
    \caption{SQUID server setup}
    \label{fig:etfw_squid_wizard_01}
    \end{center}
\end{figure}

The proxy service forward requests to the Internet and keeps a content's cache in a way to speed up the display when they are requested again.

In the \textit{SQUID Wizard} we can configure the proxy service easily. It has three pre-defined configuration options:

\begin{itemize}
    \item \textit{Transparent Proxy} - Transparent Proxy;
    \item \textit{Proxy with AD} - Proxy with authentication in Active Directory;
    \item \textit{Proxy with LDAP} - Proxy with LDAP authentication.
\end{itemize}

In the first case, the transparent proxy, allows you to have a cache system cache completely invisible to the customers. This system does not support authentication.

In the other two cases, the proxy makes use of authentication systems, such as Active Directory or LDAP.

Additionally, you can customize the service in accordance with the needs, in particular define: Ports and Networking, Access Control, Authentication Programs, among other types of cache.

\subsubsection{Ports and Networking}

In \textit{Ports and Networking} you can define the port (SSL or not), and IP address or hostname of the proxy service that will fill orders, in addition to other possible configurations, including:

\begin{figure}[H]
    \begin{center}
    \includegraphics[scale=0.38]{screenshots/etfw/etfw_squid_portsnetworking_01.png}
    \caption{Port and network configuration}
    \label{fig:etfw_squid_portsnetworking_01}
    \end{center}
\end{figure}

Request port \textit{ICP};
Validation name addresses of URLs;
Specification group \textit{multicast};
Output address of TCP traffic;
Output address UDP traffic;


\begin{itemize}
    \item \textit{ICP port} - ICP request port;
    \item \textit{Validate hostnames in URLs?} - Validation of urls' address;
    \item \textit{Multicast groups} - Multicast groups;
    \item \textit{Outgoing TCP address} - Output address for TCP traffic;
    \item \textit{Outgoing UDP address} - Output address for UDP traffic;
    \item \textit{Incoming UDP address} - Input address for UDP traffic;
    \item \textit{TCP receive buffer} - Buffer \textit{TCP};
\end{itemize}

\subsubsection{Access Control}

The access control policies are based on a combination of \textit{ACL}(\textit{Access Control Lists}).

\begin{figure}[H]
    \begin{center}
    \includegraphics[scale=0.38]{screenshots/etfw/etfw_squid_accesscontrol_01.png}
    \caption{Setup of access control policies}
    \label{fig:etfw_squid_accesscontrol_01}
    \end{center}
\end{figure}

In the configuration of access control policies we can define filtering models that can be used later in the sections of restrictions of access (\textit{Proxy Restrictions}, \textit{ICP Restrictions}, \textit{Proxy Reply Restrictions}) .

\begin{figure}[H]
    \begin{center}
    \includegraphics[scale=0.38]{screenshots/etfw/etfw_squid_accesscontrol_02.png}
    \caption{Creating a new ACL}
    \label{fig:etfw_squid_accesscontrol_02}
    \end{center}
\end{figure}

To create a new ACL, we select the type and fill with the desired parameters.

\begin{figure}[H]
    \begin{center}
    \includegraphics[scale=0.38]{screenshots/etfw/etfw_squid_accesscontrol_03.png}
    \caption{Creating a new external acl}
    \label{fig:etfw_squid_accesscontrol_03}
    \end{center}
\end{figure}

It is also possible to define external ACLs that allow to expand the functionalities of the proxy, using for this purpose external programs that manage the access.
These ACLs allow, for example, the authentication in a server such as a Active Directory or a LDAP, or even the verification of source addresses in a SQL database.

The creation of an external \textit{ACL} external requires the creation of an internal ACL with reference to the first one.

\begin{figure}[H]
    \begin{center}
    \includegraphics[scale=0.38]{screenshots/etfw/etfw_squid_accesscontrol_04.png}
    \caption{Restriction definition - \textit{Proxy restrictions}}
    \label{fig:etfw_squid_accesscontrol_04}
    \end{center}
\end{figure}

After the ACL definition, you must associate the restrictions with the ACLs to be applied in every situation, or what action to do: accept or deny.
The rules are applied in top-bottom order, and when a match is found the action takes place.
Importantly, if there is a deny all rule, all requests that pass the rules are accepted.

\subsubsection{Authentication Programs}

In \textit{Authentication Programs} are defined programs that ask the browser/user what's his authentication credentials.

\begin{figure}[H]
    \begin{center}
    \includegraphics[scale=0.38]{screenshots/etfw/etfw_squid_authenticationprograms_01.png}
    \caption{Authentication Programs}
    \label{fig:etfw_squid_authenticationprograms_01}
    \end{center}
\end{figure}

There are two types of authentication:

\begin{itemize}
    \item \textit{Basic} - when the browser does not support transparent authentication, it will show a popup asking the user credentials
    \item \textit{NTLMSSP} - transparent authentication for the user
\end{itemize}

We can even set the following parameters:

\begin{itemize}
    \item \textit{authentication program} - Specifies the user program for authentication. The program reads a line containing user and the password separated by space, and responds \textit{OK} on success or \textit{ERR} on failure;
    \item \textit{Number of authentication programs} - Maximum number of process that the authentication could have;
    \item \textit{Authentication realm} - Text that will appear in the dialog box in case of basic authentication;
    \item \textit{Authentication cache time} - Specifies how long a valid authentication is maintained avoiding requests;
    \item \textit{Number of times an NTLM challenge can be re-used} - A maximum number of times that can be used in NTLMSSP authentication type;
    \item \textit{Lifetime of NTLM challenges} - Lifetime of the \textit{NTLMSSP} authentication type;
    \item \textit{Authenticate IP cache time} - Specifies for how long the cache is kept about the user-IP association. 
\end{itemize}

\subsubsection{Other Caches}

In \textit{Other caches} we can specify other proxies to be used in a chain the get information.

\begin{figure}[H]
    \begin{center}
    \includegraphics[scale=0.38]{screenshots/etfw/etfw_squid_othercaches_01.png}
    \caption{Proxies - Other Caches}
    \label{fig:etfw_squid_othercaches_01}
    \end{center}
\end{figure}

\begin{figure}[H]
    \begin{center}
    \includegraphics[scale=0.38]{screenshots/etfw/etfw_squid_othercaches_02.png}
    \caption{Edit host cache}
    \label{fig:etfw_squid_othercaches_01}
    \end{center}
\end{figure}

To specify a different proxy it's necessary to specify the following fields:

\begin{itemize}
    \item \textit{Hostname} - IP address or hostname (\textit{FQDN}) of the cache to being used;
    \item \textit{Type} - Type hierarchy to be used between the proxies:
        \subitem \textit{parent};
        \subitem \textit{sibling};
        \subitem \textit{multicast};
    \item \textit{Proxy port} - The port that proxy uses for listening requests;
    \item \textit{ICP port} - Used port to ask neighbors about the cache objects that they have;
    \item \textit{Proxy only?} - Indicates that the requested content to this proxy is not to save locally; 
    \item \textit{Send ICP queries} - Used for proxies that does not have ICP, i.e., that indicates if they have a object or not;
    \item \textit{Default cache} - It is used when the proxy is the last in the hierarchy line;
    \item \textit{Round-robin cache} - To use the round-robin algorithm of search for proxies;
    \item \textit{ICP time-to-live} - Specifies the Time-To-Live(ttl) used in multicast protocol;
    \item \textit{Cache weighting} - Specifies the importance of the cache on the process o choosing proxy (1 for lower priority).
\end{itemize}

\subsubsection{Usage exemples}

\begin{figure}[H]
    \begin{center}
    \includegraphics[scale=0.38]{screenshots/etfw/etfw_squid_example_time_01_01.png}
    \caption{Restrict internal network access only during work hours - Creating ACLs.}
    \label{fig:etfw_squid_example_time_01_01}
    \end{center}
\end{figure}

\begin{figure}[H]
    \begin{center}
    \includegraphics[scale=0.38]{screenshots/etfw/etfw_squid_example_time_01_02.png}
    \caption{Restrict internal network access only during work hours - Creating restriction using previously defined ACLs}
    \label{fig:etfw_squid_example_time_01_02}
    \end{center}
\end{figure}

\begin{figure}[H]
    \begin{center}
    \includegraphics[scale=0.38]{screenshots/etfw/etfw_squid_example_time_02_01.png}
    \caption{Restrict access only in the morning - Creating ACLs}
    \label{fig:etfw_squid_example_time_02_01}
    \end{center}
\end{figure}

\begin{figure}[H]
    \begin{center}
    \includegraphics[scale=0.38]{screenshots/etfw/etfw_squid_example_time_02_02.png}
    \caption{Restrict access only in the morning - Creating restriction using previously defined ACLs}
    \label{fig:etfw_squid_example_time_02_02}
    \end{center}
\end{figure}

\begin{figure}[H]
    \begin{center}
    \includegraphics[scale=0.38]{screenshots/etfw/etfw_squid_example_acessoip_01.png}
    \caption{Restrict access by IP address - Creating ACLs}
    \label{fig:etfw_squid_example_acessoip_01}
    \end{center}
\end{figure}

\begin{figure}[H]
    \begin{center}
    \includegraphics[scale=0.38]{screenshots/etfw/etfw_squid_example_acessoip_02.png}
    \caption{Restrict access by IP address - Creating restriction using previously defined ACLs}
    \label{fig:etfw_squid_example_acessoip_02}
    \end{center}
\end{figure}

\begin{figure}[H]
    \begin{center}
    \includegraphics[scale=0.38]{screenshots/etfw/etfw_squid_example_urlregexp_01.png}
    \caption{Denying access based on a regular expression on the URL - Creating ACLs}
    \label{fig:etfw_squid_example_urlregexp_01}
    \end{center}
\end{figure}

\begin{figure}[H]
    \begin{center}
    \includegraphics[scale=0.38]{screenshots/etfw/etfw_squid_example_urlregexp_02.png}
    \caption{Denying access based on a regular expression on the URL - Creating restriction using previously defined ACLs}
    \label{fig:etfw_squid_example_urlregexp_02}
    \end{center}
\end{figure}

\subsection{SNMP server}

In the SNMP server configuration interface you can define the following configuration:

\begin{itemize}
    \item System information: location and contact;
    \item Trap server's IP address;
    \item Trap community;
    \item Monitoring stations.
\end{itemize}

\begin{figure}[H]
    \begin{center}
    \includegraphics[scale=0.38]{screenshots/etfw/etfw_snmp_01.png}
    \caption{SNMP server setup}
    \label{fig:etfw_smp_01}
    \end{center}
\end{figure}


%\section{ETMS}
\section{ETMS}

Através da interface do \textit{Central Management} é possível efectuar as configurações necessárias ao funcionamento do servidor de email \footnote{O sistema foi implementado de modo a manter a compatibilidade com o \textit{Webmin}.}.

A gestão divide-se em três separadores, respeitantes a diferentes contextos de configuração. O primeiro separador indica o estado do serviço e permite iniciar, parar ou reiniciar o serviço. É também possível efectuar \textit{backups} da sua configuração (no próprio agente) e o restauro da mesma - útil para testar novas configurações. No segundo separador pode ser efectuada a configuração de de domínios. Por último, o terceiro separador permite a configuração de contas de correio. As sub-secções \ref{sec:etms_gerir_dominios} e \ref{sec:etms_sub_gerir_caixas_dominio} indicam as possíveis configurações. 

A imagem \ref{fig:etms_main_info} ilustra os separadores existentes. Para os encontrar seleccione a máquina virtual que contém a instalação do \textit{ETMS}, seguido da opção \textit{ETMS} que se encontra nos separadores do painel do lado direito.


\begin{figure}[H]
    \begin{center}
    \includegraphics[scale=0.38]{screenshots/etms/etms_main_info.png}
    \caption{ETMS - Painel de Informação Principal}
    \label{fig:etms_main_info}
    \end{center}
\end{figure}


\subsection{Separador 1 - Serviço}
\label{sec:etms_info_principal}

O separador \textit{Serviço} está dividido em três colunas (imagem \ref{fig:etms_main_info}). À esquerda são indicadas informações sobre o processo que executa o serviço, nomeadamente: informação sobre o seu estado (\textit{Up} - a funcionar; \textit{Down} - parado), tempo no presente estado, o número de domínios e contas de correio existentes, e o espaço total ocupado pelos emails existentes no servidor. Note que o espaço total ocupado não fica visível logo após a abertura do separador, por se tratar de uma operação potencialmente demorada. Assim, para solicitar esta informação deve seleccionar o ícone à direita. A informação sobre o estado do serviço, é actualizada na primeira vez que o separador é aberto, podendo ser refrescada sempre que solicitada de forma explicita, através da opção \textit{Actualizar}.

Na coluna central encontram-se opções para \textit{Iniciar}, \textit{Parar} e \textit{Reiniciar} o processo que suporta o serviço. Na eventualidade de haver dificuldade em parar o serviço a opção \textit{forçar paragem} deve ser utilizada.

Na coluna à direita estão definidas as operações de \textit{Backup} e \textit{Restauro das Configurações}. Estas opções \textbf{devem apenas} ser utilizadas no teste de novas configurações, pois o seu armazenamento é efectuado localmente (na maquina onde se encontra o servidor de email).

\subsection{Separador 2 - Gerir Domínios}
\label{sec:etms_gerir_dominios}

O conteúdo do separador \textit{Gerir Domínios} é dividido em duas áreas/grelhas, onde é possível seleccionar linhas (imagem \ref{fig:etms_criar_dominio_1}). A grelha da esquerda indica quais os domínios existentes, a da direita, lista os \textit{Alias} existentes para o domínio seleccionado.

Em ambas as áreas é possível efectuar operações sobre o item seleccionado, através da utilização dos botões disponíveis na barra de ferramentas sob a grelha em questão. Note que ao seleccionar um domínio a lista de \textit{alias} é actualizada e substituída para o domínio em questão.

\begin{figure}[H]
    \begin{center}
    \includegraphics[scale=0.35]{screenshots/etms/etms_criar_dominio_1.png}
    \caption{ETMS - Painel de Gestão de Domínios}
    \label{fig:etms_criar_dominio_1}
    \end{center}
\end{figure}

\subsubsection{Criação de um Domínio}
\label{sec:etms_sub_criacao_dominio}

Para criar um domínio utilizar a opção \textit{Adicionar}, onde de seguida se abrirá uma janela com os campos a preencher, como ilustra a figura \ref{fig:etms_criar_dominio_2}. Após preencher os campos seleccionar \textit{Guardar} para efectuar a alteração. Note que: os três primeiros campos são de carácter obrigatório; a grelha com os domínios existentes é refrescada após a adição do novo domínio.

\begin{figure}[H]
    \begin{center}
    \includegraphics[scale=0.35]{screenshots/etms/etms_criar_dominio_2.png}
    \caption{Criar Domínio}
    \label{fig:etms_criar_dominio_2}
    \end{center}
\end{figure}

Para uma melhor compreensão, descrevem-se sucintamente os campos existentes, indicando para cada um exemplo:
\begin{itemize}
\item \textbf{Nome do Domínio} - Qual o nome do domínio pretendido (Ex: eurotux.com)
\item \textbf{Descrição} - Breve descrição sobre o domínio (Ex: Tecnologias de Informação)
\item \textbf{Palavra-Chave} - \textit{Password} a utilizar na gestão do domínio\footnote{Para utilizadores do Webmin}, maior que seis caracteres. (Ex: PassWord)
\item \textbf{Quota de Servidor} - Valor máximo a utilizar no armazenamento de emails(Ex: 1000000000 Bytes)
\item \textbf{Número de Caixas de Correio} - Número limite de caixas de correio que podem ser definidas para o domínio. Note que este campo pode impedir a criação de novas caixas. Porém a redução do seu valor não elimina contas previamente criadas. (Ex: 10)
\item \textbf{Activo} Indica se o serviço está activo para o domínio seleccionado. Na realidade esta opcção altera o estado das contas de email, inibindo ou não a entrega de novas mensagens.
\end{itemize}

\subsubsection{Edição de um Domínio}
\label{sec:etms_sub_edicao_dominio}
Para editar um domínio, seleccionar a linha correspondente, e escolher a opção \textbf{Editar}. Abrir-se-á uma janela (ver figura \ref{fig:etms_criar_dominio_2}) com os atributos do domínio, referidos anteriormente mas preenchidos com as configurações previamente efectuadas (subsecção anterior \ref{sec:etms_sub_criacao_dominio}). 

Após guardar, a grelha que lista os domínios existentes é actualizada com as novas configurações.

\subsubsection{Remoção de um Domínio}
\label{sec:etms_sub_remocao_dominio}
A remoção de um domínio implica também a remoção dos \textit{Alias} e contas de correio associadas (incluindo emails existentes, não sendo possível recupera-los através do processo de restauro de configurações). Para a remoção de um domínio, seleccionar a linha que identifica o domínio a remover, e escolher a opção \textbf{Apagar}. De seguida responder afirmativamente à questão que confirma a operação, como ilustra a imagem \ref{fig:etms_delete_domain}. O sucesso da operação é indicado no \textit{Painel de Informação}.

\begin{figure}[H]
    \begin{center}
    \includegraphics[scale=0.35]{screenshots/etms/etms_delete_domain.png}
    \caption{Remover Domínio}
    \label{fig:etms_delete_domain}
    \end{center}
\end{figure}


\subsection{Gerir Caixas de Correio}
\label{sec:etms_sub_gerir_caixas_dominio}
A opção \textit{Gerir Caixas de Correio} facilita a alteração das caixas de correio do domínio. Ao seleccionar esta opção é encaminhado automaticamente para o separador \textit{Gerir Caixas de Correio}, e é efectuada uma pesquisa por caixas de correio pertencentes ao domínio (ver imagem \ref{fig:etms_gerir_mailboxes}). Note que a opção \textit{Gerir Caixas de Correio} apenas fica visível se um domínio estiver seleccionado.

\begin{figure}[H]
    \begin{center}
    \includegraphics[scale=0.35]{screenshots/etms/etms_gerir_mailboxes.png}
    \caption{Gerir Caixas de Correio de um Domínio}
    \label{fig:etms_gerir_mailboxes}
    \end{center}
\end{figure}

\subsubsection{Opção Detalhes}
\label{sec:etms_sub_detalhes_dominio}
A opção \textit{Detalhes}, pertencente à barra de ferramentas que se encontra sob a \textit{Lista de Domínios} (à direita). Ao seleccionar esta opção é acrescentada uma coluna à grelha que lista os domínios, com informação sobre o espaço que cada um ocupa no sistema (ver imagem \ref{fig:etms_domain_details}). Note que, por se tratar de uma operação computacional intensiva e potencialmente demorada, ao efectuar outro tipo de operações esta coluna desaparece. Assim, sempre que se pretender actualizar/ver o espaço em disco em uso, deve utilizar-se esta opção.

\begin{figure}[H]
    \begin{center}
    \includegraphics[scale=0.35]{screenshots/etms/etms_domain_details.png}
    \caption{Espaço Ocupado pelos Domínio}
    \label{fig:etms_domain_details}
    \end{center}
\end{figure}

\subsubsection{Criação de \textit{Alias}}
\label{sec:etms_sub_criacao_alias_dominio}
Para acrescentar um \textit{alias} a um domínio deve: seleccionar o domínio ao qual se pretendem adicionar \textit{alias}; na área da direita onde consta a \textit{Lista de Alias}, escolher a opção \textit{Adicionar}. Note que é acrescentada uma entrada ao início da lista, onde pode definir o novo \textit{alias} (ver figura \ref{fig:etms_criar_alias_dominio}). Quando seleccionar noutro local o novo \textit{alias} é enviado para o agente (ficando o canto superior esquerdo a vermelho durante esta operação). Se a operação for bem sucedida, é mostrada uma notificação \ref{fig:etms_criar_alias_dominio_success}, e acrescentada uma entrada ao \textit{Painel de Informação}.

\begin{figure}[H]
    \begin{center}
    \includegraphics[scale=0.35]{screenshots/etms/etms_criar_alias_dominio.png}
    \caption{Adicionar Alias a um Domínio}
    \label{fig:etms_criar_alias_dominio}
    \end{center}
\end{figure}

\begin{figure}[H]
    \begin{center}
    \includegraphics[scale=0.35]{screenshots/etms/etms_criar_alias_dominio_success.png}
    \caption{Sucesso na Criação de Alias de Domínio}
    \label{fig:etms_criar_alias_dominio_success}
    \end{center}
\end{figure}


\subsubsection{Remoção de \textit{Alias}}
\label{sec:etms_sub_remocao_alias_dominio}
Para remover um \textit{alias} deve: seleccionar o domínio que contém o \textit{alias} (imagem \ref{fig:etms_criar_alias_dominio}); à direita, na \textit{Lista de Alias}, seleccionar o alias que se pretende remover; Escolher a opção \textit{Apagar}; Responder afirmativamente à mensagem de confirmação.

Note que, sempre que necessário, a \textit{Lista de Alias} pode ser actualizada, através da barra de ferramentas inferior.

\subsection{Separador 3 - Gerir Caixas de Correio}
\label{sec:etms_caixas_correio}
O conteúdo do separador \textit{Gerir Caixas de Correio} consiste numa área/grelha, onde cada linha corresponde a uma caixa de correio (imagem \ref{fig:etms_gerir_mailboxes_mb}). As linhas da grelha são seleccionáveis e é possível efectuar operações sobre cada selecção. Para que a grelha seja preenchida é necessário efectuar uma pesquisa de caixas de correio, que pode ser efectuada seguindo os passos indicados em \ref{sec:etms_sub_pesquisar_caixas_correio}.

\begin{figure}[H]
    \begin{center}
    \includegraphics[scale=0.35]{screenshots/etms/etms_gerir_mailboxes.png}
    \caption{ETMS - Painel de Gestão de Caixas de Correio}
    \label{fig:etms_gerir_mailboxes_mb}
    \end{center}
\end{figure}


\subsubsection{Pesquisar Caixas de Correio}
\label{sec:etms_sub_pesquisar_caixas_correio}
É possível pesquisar pelas caixas de correio de determinado domínio, bastando para isso indicar o nome do domínio na caixa que fica sob a grelha (na barra de ferramentas), e premir \textit{ENTER}. A pesquisa é então efectuada. Repare que, durante o processo de comunicação com a máquina que aloja o serviço, o icone do canto inferior direito fica animado (perto da opção \textit{Actualizar}). Caso não seja encontrado o domínio, é apresentada uma mensagem de erro. A pesquisa com sucesso de um domínio, habilita as opções para gestão das caixas de correio (ver imagem \ref{fig:etms_gerir_mailboxes_mb}).

\subsubsection{Criação de uma Caixa de Correio}
\label{sec:etms_sub_criar_caixas_correio}
Para criar uma caixa de correio é necessário fazer uma pesquisa pelo domínio (ver \ref{sec:etms_sub_pesquisar_caixas_correio}), utilizar a opção \textbf{Adicionar} (abrir-se-á uma janela com os campos a preencher, como ilustra a figura \ref{fig:etms_criar_mailbox}), após preencher os campos seleccionar \textit{Guardar} para efectuar a alteração. Note que: os três primeiros campos são de carácter obrigatório; a grelha com os domínios existentes é refrescada após a adiçao do novo domínio.

\begin{figure}[H]
    \begin{center}
    \includegraphics[scale=0.35]{screenshots/etms/etms_criar_mailbox.png}
    \caption{Criar Caixa de Correio}
    \label{fig:etms_criar_mailbox}
    \end{center}
\end{figure}

A janela de criação de uma nova caixa de correio é composta por três separadores: \textit{Opções Principais}; \textit{Alias} (ver subsecção \ref{sec:etms_sub_alias_caixas_correio}); \textit{Encaminhamento} (ver subsecção \ref{sec:etms_sub_encaminhamento_caixas_correio}).

Para uma melhor compreensão, descrevem-se sucintamente os campos existentes nas \textit{Opções Principais} indicando, sempre que oportuno, exemplos de utilização:
\begin{itemize}
\item \textbf{Conta} - Nome da conta pretendida (Ex: mfd@eurotux.com)
\item \textbf{Nome Real} - Nome do utilizador da conta \footnote{Para eventual necessidade de contacto} (Ex: Jorge Leal)
\item \textbf{Modificar Palavra-Chave} - \textit{Password} a utilizar na para aceder à caixa de correio\footnote{Para utilizadores do Webmin}, maior que seis caracteres. (Ex: PassWord)
\item \textbf{Activa} - Altera o estado da conta de email, inibindo/possibilitando a entrega de novas mensagens.
\item \textbf{Permitir Envio Externo} - Permitir à conta de email o envio de emails para fora do servidor, isto é, para domínios que não estejam definidos no servidor.
\item \textbf{Quota de Email} - Valor máximo a utilizar no armazenamento de emails. Note que à direita, a verde, é indicado o valor máximo definido para o domínio, não podendo a \textit{Quota de Email} exceder este valor (Ex: 10000 Bytes).
\item \textbf{Tipo de Entrega} - Existem quatro modos de entrega: local; encaminhado; local e encaminhado; local e encaminhado com resposta automática (ver imagem \ref{fig:etms_criar_alias_dominio_success}). Caso não seja seleccionado nenhum modo de encaminhamento, assume-se que o tipo de entrega é apenas local. Os modos de encaminhamento são descritos abaixo.
\item \textbf{Resposta Automática} - Mensagem com resposta automática, utilizada se o modo de entrega for \textit{Local com Resposta automática} (Ex: "Estou ausente")
\end{itemize}

Identificam-se os tipos de entrega possíveis:
\textit{Local} - Apenas é efectuada a entrega de emails na conta local, considerando-se também os \textit{alias} existentes para a conta.
\textit{Encaminhado} - Apenas é efectuada a entrega de emails para os emails definidos no separador \textit{Encaminhamento}.
\textit{Local e Encaminhado} - Os emails recebidos são entregues na caixa de correio local e encaminhados para os emails definidos no separador \textit{Encaminhamento}.
\textit{Local/Encaminhado com resposta automática} - Os emails são entregues na caixa local, encaminhados e é enviada uma mensagem para a origem do email, com o texto definido em \textit{Resposta Automática}.

\subsubsection{Edição de uma Caixa de Correio}
\label{sec:etms_sub_editar_caixas_correio}
A janela de edição de uma caixa de correio é semelhante à criação de novas caixas \ref{sec:etms_sub_criar_caixas_correio}, sendo que apenas é necessário seleccionar a caixa de correio que se pretende alterar, e escolher a opção \textit{Editar}. A diferença passa pelo facto do formulário ser automaticamente preenchido com as configurações da conta e o campo \textit{Modificar Palavra-Chave} fica por omissão desabilitado, sendo necessário seguir os passos em \ref{sec:etms_sub_password_caixas_correio} para proceder à sua alteração.


\subsubsection{Alterar a Palavra-Chave}
\label{sec:etms_sub_password_caixas_correio}
Para alterar a \textit{password} de determinada conta é necessário: seleccionar na grelha a conta pretendida; carregar em editar; na linha \textit{Modificar Palavra-Chave} seleccionar a caixa que lhe segue; definir a nova palavra-chave. Por último seleccionar guardar para terminar a configuração (ver imagem \ref{fig:etms_mb_pass_ed}).

\begin{figure}[H]
    \begin{center}
    \includegraphics[scale=0.35]{screenshots/etms/etms_mb_pass_ed.png}
    \caption{Alerar Palavra-Chave de Caixa de Correio}
    \label{fig:etms_mb_pass_ed}
    \end{center}
\end{figure}

\subsubsection{Definição de Alias para a Caixa de Correio}
\label{sec:etms_sub_alias_caixas_correio}
Podem ser definidos \textit{alias} para caixas de correio existentes, bastando para isso acrescentar entradas na grelha \textit{Alias} no processo de criação/edição de uma caixa de correio (descrito em \ref{sec:etms_sub_criar_caixas_correio}). Note que \textbf{deve} ser indicado o email completo (ex: mfd.alias@eurotux.com), e que as alterações têm efeito após a selecção da opção guardar (ver imagem \ref{fig:etms_alias_create}).

\begin{figure}[H]
    \begin{center}
    \includegraphics[scale=0.35]{screenshots/etms/etms_alias_create.png}
    \caption{Criar Alias para Caixa de Correio}
    \label{fig:etms_alias_create}
    \end{center}
\end{figure}

O procedimento para remover \textit{alias} passa por seleccionar o alias pretendido e escolher a opção \textit{Apagar}. Por último seleccionar \textit{Guardar} para que as alterações tenham efeito (ver imagens \ref{fig:etms_alias_mailbox_delete} e \ref{fig:etms_alias_mailbox_delete_2}).

\begin{figure}[H]
    \begin{center}
    \includegraphics[scale=0.35]{screenshots/etms/etms_alias_mailbox_delete.png}
    \caption{Eliminar Alias de uma Caixa de Correio - passo 1}
    \label{fig:etms_alias_mailbox_delete}
    \end{center}
\end{figure}

\begin{figure}[H]
    \begin{center}
    \includegraphics[scale=0.35]{screenshots/etms/etms_alias_mailbox_delete_2.png}
    \caption{Eliminar Alias de uma Caixa de Correio - passo 2}
    \label{fig:etms_alias_mailbox_delete_2}
    \end{center}
\end{figure}

\subsubsection{Definição de Caixas de Correio para Encaminhamento}
\label{sec:etms_sub_encaminhamento_caixas_correio}
Podem ser definidas caixas de correio para as quais os emails são encaminhados, bastando para isso acrescentar entradas na grelha \textit{Encaminhamento} no processo de criação/edição, descrito em \ref{sec:etms_sub_criar_caixas_correio}, de uma caixa de correio. Note que \textbf{deve} ser indicado o email completo (ex: mfd@eurotux.pt). As alterações têm efeito após a selecção da opção guardar (ver imagem \ref{fig:etms_alias_create}).

\begin{figure}[H]
    \begin{center}
    \includegraphics[scale=0.35]{screenshots/etms/etms_forwarding_mb_del.png}
    \caption{Definir Endereços para Encaminhamento de Emails}
    \label{fig:etms_forwarding_mb_del}
    \end{center}
\end{figure}

A remoção de caixas de encaminhamento é análogo à remoção de alias, bastando seleccionar o email em questão e escolher a opção \textit{Apagar}. Seleccionar \textit{Guardar} para que as alterações tenham efeito (procedimento análogo ao descrito em \ref{sec:etms_sub_alias_caixas_correio}).

\subsubsection{Caixas de Correio Disponíveis}
\label{sec:etms_sub_disponiveis_caixas_correio}
No caso em que o domínio pesquisado possua um limite de número de caixas de correio, aparece sobre o canto direito do separador, o número de caixas que ainda pode ser criado (ver imagem \ref{fig:etms_free_mb}). Este valor sofre um decremento após a criação de novas caixas, desabilitando a opção que possibilita a criação de novas caixas de correio quando o número de caixas de correio iguala/excede o limite definido.

\begin{figure}[H]
    \begin{center}
    \includegraphics[scale=0.4]{screenshots/etms/etms_free_mb.png}
    \caption{Caixas de Correio Disponveis}
    \label{fig:etms_free_mb}
    \end{center}
\end{figure}


\subsubsection{Opção Detalhes}
\label{sec:etms_sub_detalhes_caixas_correio}
A opção \textit{Detalhes}, pertencente à barra de ferramentas que se encontra sob a \textit{Lista de Caixas de Correio} (à direita). Ao seleccionar esta opção são acrescentadas duas colunas à grelha que lista as caixas de correio, com informação relativa ao espaço ocupado pelas mensagens recebidas lidas e por ler (ver imagem \ref{fig:etms_mb_space}). Note que, por se tratar de uma operação computacional intensiva e potencialmente demorada, ao efectuar outro tipo de operações a coluna desaparece. Assim, sempre que se pretender actualizar/ver o espaço em disco ocupado, deve utilizar-se esta opção. No caso em que novas caixas de correio sejam acrescentas, o valor não aparece, tal acontece devido ao facto das directorias que armazenam os emails ainda não terem sido criadas.

\begin{figure}[H]
    \begin{center}
    \includegraphics[scale=0.35]{screenshots/etms/etms_mb_space.png}
    \caption{Espaço Ocupado pelos Email da Caixas de Correio}
    \label{fig:etms_mb_space}
    \end{center}
\end{figure}



\subsubsection{Remoção de uma Caixa de Correio}
\label{sec:etms_sub_apagar_caixas_correio}
A remoção de uma caixa de correio, para além de remover todas as configurações, elimina todos os emails existentes nessa conta, não sendo possível recupera-los através do processo de restauro de configurações. Para proceder à remoção de conta, efectuar a pesquisa pelas caixas de correio do domínio \ref{sec:etms_sub_pesquisar_caixas_correio}, seleccionar a linha a que corresponde o caixa de correio, escolher a opção apagar e responder afirmativamente à mensagem de confirmação (imagem \ref{fig:etms_mb_del}).

\begin{figure}[H]
    \begin{center}
    \includegraphics[scale=0.35]{screenshots/etms/etms_mb_del.png}
    \caption{Confirmar Eliminação da Caixa de Correio}
    \label{fig:etms_mb_del}
    \end{center}
\end{figure}

\pagebreak
%\section{ETVOIP}
\section{ETVOIP}
In the ETVOIP tab of a virtual machine that, has the solution installed, we are enabled to manage a VOIP solution ETVOIP.
Currently ETVOIP is the interaction with the PBX component and may interact with other future developments.
The available modules for this agent are:

\begin{itemize}
    \item Extensions
    \item Trunks
    \item Outbound routes
    \item Inbound routes
\end{itemize}


\begin{quote}
	{\large \bf Note} \\*[-.8pc]
	\underline{\hspace{6in}} \\
    At the end of all operations/changes you should use the \emph{Apply changes} button available in any of the modules, in order to reflect these changes in the current configuration of the VOIP system.
\end{quote}


\begin{figure}[H]
        \begin{center}
        \includegraphics[scale=0.45]{screenshots/etvoip_pbx.png}
        \caption{Main ETVOIP panel management}
        \label{fig:etvoip_pbx}
        \end{center}
\end{figure}

In addition to these modules there is also the option to open a window FREEPBX alone, having access to advanced settings (Menu FREEPBX).

\subsection{Extensions}

In extensions pane it's possible to add, edit an remove extensions.

\begin{quote}
	{\large \bf Note} \\*[-.8pc]
	\underline{\hspace{6in}} \\
    You can only create/edit SIP extensions \footnote{SIP is a standard protocol designed for VoIP devices} and/or IAX\footnote {IAX it's the protocol "Inter Asterisk" used to interconnect asterisk servers.}.
\end{quote}

\begin{figure}[H]
        \begin{center}
        \includegraphics[scale=0.45]{screenshots/etvoip_pbx_extensions.png}
        \caption{Extension management panel}
        \label{fig:etvoip_pbx_extensions}
        \end{center}
\end{figure}


\subsubsection{Add extension}
\label{sec:etvoip_pbx_extensions_add}
When creating an extension is possible to choose between the basic/advanced settings view.

\begin{figure}[H]
        \begin{center}
        \includegraphics[scale=0.6]{screenshots/etvoip_pbx_extensions_sip.png}
        \caption{Add extension window (SIP)}
        \label{fig:etvoip_pbx_extensions_sip}
        \end{center}
\end{figure}


\begin{description}
	\item[Advanced mode: OFF -] This mode is available only the basic fields to create an extension.
        \begin{description}
            \item[Extension -] Extension parameters.
                \begin{itemize}
                    \item User extension - User extension number. Must be unique.
                    \item Display name - User name shown on his calls. Introduce a name, not a number.
                \end{itemize}

            \item[Device options -] Options for the type of device chosen.
                \begin{itemize}
                    \item Secret - Password that devices must use to authenticate on the asterisk server.
                    \item Dtmfmode - Multi Dual-Tone frequency.
                        \begin{itemize}
                            \item inband - The sending device generates DTMF tones.
                            \item outband - The ring tones are removed from the audio data and sent from a different channel.
                            \item rfc2833 - Specify a format for sending RTP packets, in order to reduce the transmitted data. Used by default.
                        \end{itemize}
                \end{itemize}
        \end{description}
	\item[Advanced mode: ON -] In this mode beyond the parameters mentioned above you can configure the following:
        \begin{description}
            \item[Extension -] Configuration parameters for extension configuration.
                \begin{itemize}
                    \item Alternative CID number - CID number to be used for internal calls, if different from the extension number. Used to masquerade as a different user. A common example is when a support team need to have your internal caller ID to show the overall number of support. Has no effect on incoming calls.
                    \item Alternative SIP - If you want to support direct sip calls to internal users or through anonymous sip calls, can provide a friendly name that can be used instead of the user's extension.
                \end{itemize}

            \item[Extension options -] Advanced extension options.
                \begin{itemize}
                    \item Outbound CID - Replaces the caller ID if passes a trunk. Overrides the outbound CID of the trunk.
                    \item Ring time - Number of seconds ringing before sending the caller to voicemail. If you do not have voicemail set this parameter is ignored.
                    \item Call waiting - Sets the initial state of call waiting for this extension 
                    \item Call screening - Requires the caller to say his name, which is then heard by the user, allowing him to accept or reject the call. Memory Screening (\emph{Screen Caller:Memory}) checks only once the source of the caller ID. Screening without memory (\emph{Screen Caller: No Memory}) always requires that the external user say his name. Either way will announce whenever the user from the last entry saved with this caller ID.
                    \item Pinless dialing - Enabled allows you to make outgoing calls from this extension without dialing PIN.
                    \item Emergency CID - This caller ID is used whenever you select an exit route marked out as an emergency. The Emergency CID overrides all other settings the caller ID.
                \end{itemize}

            \item[Assigned DID/CID -] Definition of the incoming route for this extension.
                \begin{itemize}
                    \item DID description - Incoming route description.
                    \item Add inbound DID - Sets the incoming number associated with this extension (Direct Inward Dialing).
                    \item Add inbound CID - Allows you to specify a best route DID + CID. DID should be specified in the above parameter.
                \end{itemize}

            \item[Voicemail \& Directory -] Voicemail parameters.
                \begin{itemize}
                    \item Status - Enable/disable this voicemail extension.
                    \item Voicemail Password - Password to access the voicemail system. The password can only contain numbers. A user can change the password after accessing the voicemail system (*98) in his voip phone.
                    \item Email Address - Email address of destination where voicemail notifications are sent.
                    \item Pager Email Address - Email address (pager/mobile) phone for sending  voicemail notifications.
                    \item Email Attachment - Allows you to attach voicemails to email.
                    \item Play CID - Read the phone number of origin before playing the incoming message.
                    \item Play Envelope - Read the date/time of the message.
                    \item Delete Voicemail - If enabled the message will be deleted from voicemail (after AIDS mailed). Allows a user to get his voicemail via email, without having to retrieve voicemail via the web interface or phone. CAUTION: It needs to have voicemail attached to the email, otherwise the messages will be lost.
                    \item IMAP Username - IMAP username, if in use.
                    \item IMAP Password - IMAP password.
                    \item VM Options - Extra voicemail options separated by | (such as review = yes | maxmessage = 60).
                    \item VM Context - Context used by the voicemail system. Use 'default' if you do not know the implications.
                \end{itemize}

            \item[Dictation services -] Parameters for diction service. If enabled, allows the user to dial *34 from his phone and record the conversation. The speech will be record in the defined format and sent to the specified e-mail.
                \begin{itemize}
                    \item Dictation Device
                    \item Dictation Format
                    \item Email Address
                \end{itemize}

            \item[Language -] Parameters of the extension language.
                \begin{itemize}
                    \item Language Code - If installed, it will ask if the user want to use the selected language.
                \end{itemize}

            \item[Recording options -] Extension recording parameters.
                \begin{itemize}
                    \item Record Incoming - Record all incoming calls from this extension.
                    \item Record Outgoing - Record all outgoing calls from this extension.
                \end{itemize}
            
        \end{description}
\end{description}

\subsubsection{Edit an extension}

To edit an extension is necessary to select the desired extension and click on button \emph{Edit extension}. Then you will see a window (see Figure \ref{fig:etvoip_pbx_extensions_sip}) filled with the extension's settings.
The parameters are identical to those provided in section \ref{sec:etvoip_pbx_extensions_add}.

\subsubsection{Remove an extension}
Select the extension to be removed and then click on \emph{Remove extension} button.
A window will appear confirming the removal (Figure \ref{fig:etvoip_pbx_extensions_remove}). After removing the extension and if you do not want to perform any other operation, you must apply the changes made in button \emph{Apply changes}.

\begin{figure}[H]
        \begin{center}
        \includegraphics[scale=0.6]{screenshots/etvoip_pbx_extensions_remove.png}
        \caption{Remove an extension}
        \label{fig:etvoip_pbx_extensions_remove}
        \end{center}
\end{figure}

\subsection{Trunks}

In the section \emph{Trunks} you do add/edit and remove operations.

\begin{quote}
	{\large \bf Note} \\*[-.8pc]
	\underline{\hspace{6in}} \\
    It's only possible to create/edit SIP/IAX trunks.
\end{quote}

\begin{figure}[H]
        \begin{center}
        \includegraphics[scale=0.45]{screenshots/etvoip_pbx_trunks.png}
        \caption{Trunk management panel}
        \label{fig:etvoip_pbx_trunks}
        \end{center}
\end{figure}

\subsubsection{Add Trunk}
\label{sec:etvoip_pbx_trunks_add}
When creating a trunk you can choose between the basic or advanced view by pressing the top left button.

\begin{figure}[H]
        \begin{center}
        \includegraphics[scale=0.45]{screenshots/etvoip_pbx_trunks_iax.png}
        \caption{IAX2 trunk creation window}
        \label{fig:etvoip_pbx_trunks_iax}
        \end{center}
\end{figure}


\begin{description}
	\item[Advanced mode: OFF -] This mode only show the basic fields needed in order to create a trunk.
        \begin{description}
            \item[General settings -] Trunk parameters.
                \begin{itemize}
                    \item Trunk description - Name that better describes the trunk.
                    \item Outbound CID - Call identifier used in calls made through this trunk.
                \end{itemize}

            \item[Incoming setting -] Input configurations.
                \begin{itemize}
                    \item USER Context - This is usually the account name or number that the provider is waiting. This USER Context is used to define the details of the user outlined below.
                    \item USER Details - User connection parameters to the VOIP system.
                \end{itemize}
            \item[Outgoing setting -] Output configurations.
                \begin{itemize}
                    \item Trunk name - Unique name of the trunk.
                    \item PEER details - Connection parameters between the PEER and the VOIP system.
                \end{itemize}
            \item[Registration -] VOIP registration parameters.
                \begin{itemize}
                    \item Register string - Many VoIP providers require the system to REGISTER. Enter the online registration here (example: username:password@switch.voipprovider.com). Many providers require a DID number (ex: username:password@switch.voipprovider.com/didnumber) to work the match DID
                \end{itemize}
        \end{description}
	\item[Advanced mode: ON -] This mode shows the available advanced parameters.
        \begin{description}
            \item[General settings -] Advanced trunk parameters.
                \begin{itemize}
                    \item CID Options - Determines what CIDs are allowed on this trunk. IMPORTANT: CIDs emergency defined in an extension will always be used if the trunk is part of an emergency route regardless of the settings.
                        \begin{itemize}
                            \item Allow Any CID - All CIDs including those from external forwarded calls will be transmitted.
                            \item Block Foreign CIDs - Blocks forwarded CIDs. The defined CIDs will be transmitted.
                            \item Remove CNAM - This option will remove CNAM from each CID sent by this trunk.
                            \item Force Trunk CID - Always use the CID set to trunk unless it is part of an emergency route to an emergency CID set to an extension.
                        \end{itemize}                         
                    \item Maximum Channels - Controls the maximum number of output channels (simultaneous calls) that can be made in this trunk. Incoming calls are not considered. Leave blank to not specify a limit.
                    \item Disable Trunk - Disable the use of this trunk in all routes where it is used.
                    \item Monitor Trunk Failures - If enabled, enter the name of an AGI script to be used for logging, email, or perform any action in case of failure, if the failures are not caused by NOANSWER or CANCEL.
                \end{itemize}

            \item[Outgoing dial rules -] Advanced dial options.
                \begin{itemize}
                    \item Dial Rules - A dialing rule controls how calls will be marked in this trunk. It can be used to add or remove prefixes. If the numbers didn't match the standards setted here, they will be dialled with no change. A pattern without + or | (to add or remove a prefix) will not make changes but will create a match. Only the first match found will be performed:
                        \begin{itemize}
                            \item X - Pattern matching with digits from 0 to 9.
                            \item Z - Pattern matching with digits from 1 to 9.
                            \item N - Pattern matching with digits from 2 to 9.
                            \item [1237-9] - Pattern matching with number or letters between brackets (e.g. 1,2,3,7,8,9).
                            \item . - Wildcard, pattern matching with one or more chars (not allowed before | or +).
                            \item | - Removes a dial prefix (e.g., 613|NXXXXXX matches when someone dial "6135551234", but only "5551234" passes into the trunk).
                            \item + - Adds a dial prefix into the number (for example, 1613+NXXXXXX matches when someone dial "5551234", but only "16135551234" is passed into the trunk).
                        \end{itemize}
                        We can use, simultaneously, + and |, for example: 01+0|1ZXXXXXXXXX does match with "016065551234" and marks it as "0116065551234". Note that the order does not matter, ie, 0|01+1ZXXXXXXXXX does exactly the same thing.
                    \item Outbound Dial Prefix - The outbound prefix is used to put a prefix on all outgoing calls from this trunk. For example, if the trunk has behind another PBX, we can use 9 to access an outgoing line. Most users should leave the option blank.
                \end{itemize}
        \end{description}
\end{description}


\subsubsection{Edit Trunk}

To edit a trunk we need to select the trunk to remove and click on option \emph{Edit trunk}. Then we see a window (see Figure \ref{fig:etvoip_pbx_trunks_iax}) filled with the definitions of the trunk.
The parameters are identical to those provided in section \ref{sec:etvoip_pbx_trunks_add}.

\subsubsection{Remove trunk}

To remove a selected trunk click on \emph{Remove trunk} option.
A window will appear confirming removal of the trunk (Figure \ref{fig:etvoip_pbx_trunks_remove}). After removal of the trunk and if they do not want to perform any other operation, you must apply the changes made - \emph{Apply changes} option.

\begin{figure}[H]
        \begin{center}
        \includegraphics[scale=0.6]{screenshots/etvoip_pbx_trunks_remove.png}
        \caption{Remove trunk}
        \label{fig:etvoip_pbx_trunks_remove}
        \end{center}
\end{figure}


\subsection{Outbound routes}
The \emph{Outbound routes} configures the behavior of outgoing calls. The dialed number is analyzed, and a pattern matching is made in order to find the outbound route. After that the call is forwarded to the respective trunk.

You can add, edit and remove exit routes.

\begin{figure}[H]
        \begin{center}
        \includegraphics[scale=0.45]{screenshots/etvoip_pbx_outbound_routes.png}
        \caption{Outbound management panel}
        \label{fig:etvoip_pbx_outbound_routes}
        \end{center}
\end{figure}

\subsubsection{Add route}
\label{sec:etvoip_pbx_outbound_routes_add}

When you select the option \emph{Add route} a window will appear (see Figure \ref{fig:etvoip_pbx_outbound_routes_add}) where you can choose between the basic/advanced settings view setting mode. The following parameters are available:

\begin{description}
	\item[Advanced mode: OFF -] In this mode you can define the route basic settings.
        \begin{description}
            \item[General settings -] Route parameters.
                \begin{itemize}
                    \item Route Name - Route name (e.g., 'local').
                    \item Dial Patterns - Pattern that the dialing number must match to select this route:
                        \begin{itemize}
                            \item X - Pattern matching with digits from 0 to 9.
                            \item Z - Pattern matching with digits from 1 to 9.
                            \item N - Pattern matching with digits from 2 to 9.
                            \item [1237-9] - Pattern matching with number or letters between brackets (e.g. 1,2,3,7,8,9).
                            \item . - Wildcard, pattern matching with one or more chars (not allowed before | or +).
                            \item | - Removes a dial prefix (e.g., 613|NXXXXXX matches when someone dial "6135551234", but only "5551234" passes into the trunk).
                            \item / - Adds to the pattern. Does the match with a CID or pattern (e.g., NXXXXXX/104 matches only with the "104" extension).
                        \end{itemize}
                \end{itemize}

            \item[Trunks -] Sequence output trunks. The sequence of trunks control the order of trunks that will be used when the number matches the defined patterns.
        \end{description}
	\item[Advanced mode: ON -] This mode shows some additional configurations that can be made:
        \begin{description}
            \item[General settings -] Trunk parameters.
                \begin{itemize}
                    \item Route CID - If selected will re-write all specific CIDs except:
                        \begin{itemize}
                            \item Emergency CIDs (extension/device) if this route was marked as an emergency route.
                            \item Trunk CID if the trunk is marked to force the CID.
                            \item CIDs from forwarded calls (CF, Follow Me, Ring Groups, etc).
                            \item CIDs from extensions/users if enabled.
                        \end{itemize}                       
                    \item Route Password - The route can request the user to insert a password before allowing the call. Useful in cases of international call barring.
                       A numerical password, or the path to a file with the password to be used. Leave this field blank unless required for a password.
                    \item Emergency Dialing - Selecting this option will force the use of an emergency CID (if configured). This option allows that a given set of routes must been used to dial the emergency number (eg: 112).
                    \item Internal Company Route - By selecting this option the route will be treated as an intra-company connection, preserving the internal CID information and not using the output CID either the extension or trunk.
                    \item Music On Hold? - Defines the type of music when the call is on hold.
                \end{itemize}            
        \end{description}
\end{description}


\begin{figure}[H]
        \begin{center}
        \includegraphics[scale=0.45]{screenshots/etvoip_pbx_outbound_routes_add.png}
        \caption{Add outbound route window}
        \label{fig:etvoip_pbx_outbound_routes_add}
        \end{center}
\end{figure}


\subsubsection{Edit route}
To edit an outbound route is necessary to select the desired route and click on \emph{Edit route}. Then you will see a window (see Figure \ref{fig:etvoip_pbx_outbound_routes_add}) filled with the definitions of the route.
The parameters are identical to those provided in section \ref{sec:etvoip_pbx_outbound_routes_add}.

\subsubsection{Remove route}
To remove a selected outbound route click on the \emph{Remove route} button.
A window will appear confirming the removal of the route (Figure \ref{fig:etvoip_pbx_outbound_routes_remove}). After the removal and if you do not want to perform any other operation, you must apply the changes made pressing the \emph{Apply changes} button.

\begin{figure}[H]
        \begin{center}
        \includegraphics[scale=0.6]{screenshots/etvoip_pbx_outbound_routes_remove.png}
        \caption{Remove outbound route}
        \label{fig:etvoip_pbx_outbound_routes_remove}
        \end{center}
\end{figure}


\subsection{Incoming routes}
In \emph{Inbound Routes} we can configure the behavior of incoming calls of all trunks.
When receiving an incoming call, the VOIP server needs to know where to redirect it.
Can be redirected to a Ring Group, an extension or IVR, among other options.

Therefore, it is possible to perform operations to add, edit and remove routes of entry.
In the management panel you can view the description and the numbers DID/CID associated with each route.

\begin{figure}[H]
        \begin{center}
        \includegraphics[scale=0.45]{screenshots/etvoip_pbx_inbound_routes.png}
        \caption{Inbound routes management panel}
        \label{fig:etvoip_pbx_inbound_routes}
        \end{center}
\end{figure}

\subsubsection{Add route}
\label{sec:etvoip_pbx_inbound_routes_add}

When you select \emph{Add route}, a window will appear (see Figure \ref{fig:etvoip_pbx_inbound_routes_add}) where you can set the following parameters:

\begin{description}
            \item[General Settings -] General route parameters.
                \begin{itemize}
                    \item \textbf{Description} - Description of the route.
                    \item \textbf{DID Number} - Expected DID number, if the DID trunk accept incoming calls. Leave it blank to match all DID.
                    \item \textbf{Caller ID Number} - Sets the CID to match the incoming calls. Leave it blank to match all the ICD.
                \end{itemize}

            \item[Set Destination -] Destination of calls that match the DID/CID number.
                \begin{itemize}
                    \item \textbf{Ring Groups} - Extension group.
                    \item \textbf{Terminate Call} - The call is automatically ended.
                    \item \textbf{Phonebook Directory} - The contact list is shown.
                    \item \textbf{IVR}\footnote{Acronym for \emph{Interactive Voice Response}} - virtual receptionist.
                    \item \textbf{Extensions} - Pre-defined extension.
                \end{itemize}
       
\end{description}

\begin{figure}[H]
        \begin{center}
        \includegraphics[scale=0.6]{screenshots/etvoip_pbx_inbound_routes_add.png}
        \caption{Inbound route creation window}
        \label{fig:etvoip_pbx_inbound_routes_add}
        \end{center}
\end{figure}

\subsubsection{Edit route}
To edit a route you must select the desired route and click on \emph{Edit route} option. Then, you will see a confirmation window (see Figure \ref{fig:etvoip_pbx_inbound_routes_add}) filled with the definitions of the route.
The parameters are identical to those provided in section \ref{sec:etvoip_pbx_inbound_routes_add}.

\subsubsection{Remover rota}
To remove a selected route entry click on \emph{Remove route} option.
A window will appear confirming the removal (Figure \ref{fig:etvoip_pbx_inbound_routes_remove}). After removal and if you do not want to perform any other operation, you must apply the changes made - \emph{Apply changes} option.

\begin{figure}[H]
        \begin{center}
        \includegraphics[scale=0.6]{screenshots/etvoip_pbx_inbound_routes_remove.png}
        \caption{Remove inbound route}
        \label{fig:etvoip_pbx_inbound_routes_remove}
        \end{center}
\end{figure}


\pagebreak

%\section{Primavera}
\section{Primavera}

O \textit{UnitBox} disponibiliza no \textit{Central Management} um separador que permite editar as configuração relativas aos serviços do \textit{Primavera}.

\subsection{Instalação}

Para activar o separador do \textit{Primavera} no \textit{Central Management} é necessário proceder à instalação do agente na máquina virtual.

O processo de instalação segue os seguintes passos:

\begin{figure}[H]
    \begin{center}
    \includegraphics[scale=0.38]{screenshots/primavera/primaverainstall_01.png}
    \caption{Passo 1 - Escolha da directoria de instalação}
    \label{fig:primavera_install_passo1}
    \end{center}
\end{figure}

\begin{figure}[H]
    \begin{center}
    \includegraphics[scale=0.38]{screenshots/primavera/primaverainstall_02.png}
    \caption{Passo 2 - Estado da instalação}
    \label{fig:primavera_install_passo2}
    \end{center}
\end{figure}

\begin{figure}[H]
    \begin{center}
    \includegraphics[scale=0.38]{screenshots/primavera/primaverainstall_03.png}
    \caption{Passo 3 - Parâmetros de configuração do Agente}
    \label{fig:primavera_install_passo3}
    \end{center}
\end{figure}

No passo 3, é necessário configurar alguns parâmetros que vão permitir aceder aos motores do \textit{Primavera} e assim permitir a sua gestão.

\begin{figure}[H]
    \begin{center}
    \includegraphics[scale=0.38]{screenshots/primavera/primaverainstall_04.png}
    \caption{Passo 4 - Confirmação de reboot}
    \label{fig:primavera_install_passo4}
    \end{center}
\end{figure}

\begin{figure}[H]
    \begin{center}
    \includegraphics[scale=0.38]{screenshots/primavera/primaverainstall_05.png}
    \caption{Passo 5 - Inicialização do agente}
    \label{fig:primavera_install_passo5}
    \end{center}
\end{figure}

Após instalação é necessário proceder ao \textit{reboot} da máquina virtual para inicializar o agente.
A partir deste momento será possível a gestão do \textit{Primavera} a partir da interface do \textit{Central Mangement}.

\subsection{Interface}

A interface de gestão do \textit{Primavera} permite aceder algumas funcionalidades essenciais para a manutenção do serviço, nomeadamente, gestão de \textit{backups}, parar/arrancar serviço, gestão de utilizadores e alteração do endereço do IP da máquina virtual.

\begin{figure}[H]
    \begin{center}
    \includegraphics[scale=0.38]{screenshots/primavera/primaverainterface_01.png}
    \caption{Informação do serviço}
    \label{fig:primavera_info}
    \end{center}
\end{figure}

Logo ao aceder ao separador do \textit{Primavera} é apresentado informação relativa ao estado do serviço: utilização de espaço em disco, número de empresas, licença do \textit{Primavera}, número de postos, estado dos serviços \textit{Primavera} e \textit{SQL Server}, e informação de rede.

A partir da barra de menu é possível aceder a outras funcionalidades, nomeadamente, \textit{Backup} e restauro, parar/iniciar serviço \textit{Primavera}, alterar configuração de rede e gestão de utilizadores.

\begin{figure}[H]
    \begin{center}
    \includegraphics[scale=0.38]{screenshots/primavera/primaverainterface_02.png}
    \caption{Lista de planos de backup}
    \label{fig:primavera_list_backup_plans}
    \end{center}
\end{figure}

No menu \textit{Backup} acedemos à funcionalidade que permite gerir os planos de backup.
Para criar um novo plano de \textit{backup}, acedemos ao separador \textit{Novo plano de backup} e onde somos obrigados a definir os seguintes parâmetros: \textit{Nome} - identificação do plano, \textit{Período} - periodicidade do plano de backup (diário, semanal, mensal), \textit{Base de dados} - base de dados que serão efectuadas \textit{backup}, opções \textit{Verificar}, \textit{Sobrescrever} e \textit{Incremental}, configuram o plano para verificar após efectuar \textit{backup}, sobre-pôr o ficheiro em caso de já existir e \textit{backup} incremental.

\begin{figure}[H]
    \begin{center}
    \includegraphics[scale=0.38]{screenshots/primavera/primaverainterface_03.png}
    \caption{Criar novo plano de backup}
    \label{fig:primavera_new_backup_plan}
    \end{center}
\end{figure}

\begin{figure}[H]
    \begin{center}
    \includegraphics[scale=0.38]{screenshots/primavera/primaverainterface_04.png}
    \caption{Efectuar backup na hora}
    \label{fig:primavera_backup_now}
    \end{center}
\end{figure}

Existe ainda a opção para efectuar um \textit{backup} de imediato no separador \textit{Backup agora}. 
Aqui podemos escolher a empresa a efectuar backup ou então efectuar um \textit{backup} completo à plataforma \textit{Primavera}, escolhendo a opção \textit{Backup integral}.

\begin{figure}[H]
    \begin{center}
    \includegraphics[scale=0.38]{screenshots/primavera/primaverainterface_05.png}
    \caption{Efectuar restauro}
    \label{fig:primavera_restore}
    \end{center}
\end{figure}

No separador de \textit{Restauro} acedemos à funcionalidade que permite repor um \textit{backup}. 
Podemos efectuar a reposição de um \textit{backup} para uma determinada empresa ou então efectuar o restauro completo do sistema com o último \textit{backup} ( opção \textit{Restauro integral} ).

\begin{figure}[H]
    \begin{center}
    \includegraphics[scale=0.38]{screenshots/primavera/primaverainterface_06.png}
    \caption{Alterar IP}
    \label{fig:primavera_change_ip}
    \end{center}
\end{figure}

Para efectuar alteração da configuração de rede podemos aceder ao separar \textit{Alterar IP}, onde podemos modificar os parâmetros de \textit{Endereço IP}, \textit{Máscara de rede} e \textit{Gateway} da máquina virtual do \textit{Primavera}. É também possível definir que esta configuração é atribuída por \textit{DHCP}.

\begin{figure}[H]
    \begin{center}
    \includegraphics[scale=0.38]{screenshots/primavera/primaverainterface_07.png}
    \caption{Lista de utilizadores Primavera}
    \label{fig:primavera_list_users}
    \end{center}
\end{figure}

No separador \textit{Utilizadores} acedemos à interface de gestão dos utilizadores \textit{Primavera}.
Aqui é possível consultar os utilizadores existentes, adicionar novo utilizador com os vários parâmetros (Utilizador, Nome, Email, Password e opções Super-Administrador, Administrador e/ou Técnico ). É ainda possível editar os dados de um utilizador ou até mesmo remover do sistema.

\begin{figure}[H]
    \begin{center}
    \includegraphics[scale=0.38]{screenshots/primavera/primaverainterface_08.png}
    \caption{Adicionar utilizador Primavera}
    \label{fig:primavera_add_user}
    \end{center}
\end{figure}

\begin{figure}[H]
    \begin{center}
    \includegraphics[scale=0.38]{screenshots/primavera/primaverainterface_09.png}
    \caption{Editar utilizador Primavera}
    \label{fig:primavera_edit_user}
    \end{center}
\end{figure}

\begin{figure}[H]
    \begin{center}
    \includegraphics[scale=0.38]{screenshots/primavera/primaverainterface_10.png}
    \caption{Remover utilizador Primavera}
    \label{fig:primavera_delete_user}
    \end{center}
\end{figure}


