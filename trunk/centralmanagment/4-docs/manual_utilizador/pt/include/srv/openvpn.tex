\subsection*{VPN-OpenVPN}

Os utilizadores de VPN, bem como os IP que lhes serão atribuídos, podem ser
configurados no ficheiro \emph{/etc/ppp/chap-secrets}.
Não é necessário reiniciar o serviço de VPN para adicionar novos utilizadores ou para alterar os existentes.

Na interface \textit{web} também existe a possibilidade de se criarem, removerem e alterarem
utilizadores (conforme se pode observar na Figura~\ref{fig:vpn}), bastando seguir a opção ``Rede'' $\rightarrow$ ``Servidor VPN OpenVPN'' na árvore de navegação horizontal.

\begin{figure}[H]
\begin{center}
\includegraphics[width=15cm]{include/img/vpn}
\end{center}
\caption{Edição de utilizadores da VPN}
\label{fig:vpn}
\end{figure}

Este serviço é iniciado automaticamente no arranque da plataforma mas pode ser desligado utilizando o seguinte comando via SSH ou na linha de comandos:

\begin{verbatim}
# service openvpn stop
\end{verbatim}

Pode ser iniciado manualmente através do comando:

\begin{verbatim}
# service openvpn start
\end{verbatim}
