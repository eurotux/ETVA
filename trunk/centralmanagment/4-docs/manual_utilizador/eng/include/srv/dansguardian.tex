\subsection{Serviço de Filtragem de Conteúdo -- Dansguardian}

\subsubsection{Configuração}

O servidor de filtragem de conteúdo tem a sua configuração em
/etc/dansguardian pelo que futuras referências farão uso desta
mesma informação.

\confoption{dansguardian.conf}~\\

O principal ficheiro de configuração é o \emph{dansguardian.conf} e
contém as características principais de funcionamento nomeadamente as
portas onde está à espera de pedidos, para que servidor envia os
pedidos e quais os ficheiros que especificam as regras de filtragem.
É possível também definir qual o número máximo e mínimo de processos
de filtragem usando as directivas \emph{maxchildren},
\emph{minchildren}, \emph{minsparechildren}, \emph{maxsparechildren}
e \emph{maxagechildren} tal como estão comentadas no ficheiro em
questão.
De seguida são apresentados alguns parâmetros de configuração:

\begin{description}
\item[Reporting Level]~\\
Permite modificar o nível de verbosidade quando uma página é negada.
Pode dizer apenas \emph{Access Denied}, ou explicar porquê, ou então
qual a frase negada.

\item[Logging Settings]~\\
Permite configurar o nível de \emph{logging}.

\item[Log File Format]~\\
Permite mudar o formato de como o dansguardian guarda os \emph{logs}.

\item[Network Settings]~\\
Permite modificar a porta onde o dansguardian está à espera de pedidos,
o endereço IP do servidor que tem o squid bem como a respectiva porta. 
Permite configurar a página que irá ser mostrada no caso de o acesso
for negado.

\item[Content Filtering Settings]~\\
Permite mudar o local onde estão os ficheiros com conteúdos filtrados.

\item[Naughtyness limit]~\\
Esta directiva permite definir o limite a partir do qual um conteúdo
irá ser bloqueado.

Cada palavra ou conjunto de palavras poderá ter um peso positivo
(ou seja, vai levar a que o conteúdo seja mais rapidamente bloqueado)
ou um peso negativo.

O ficheiro \emph{weightedphraselist} contém alguns exemplos.
Os seguintes valores são usados frequentemente, 50 para crianças, 100
para jovens e 160 para adultos.

\item[Show weighted phrases found]~\\
Se ligado, as palavras ou frases encontradas que levam a que a
pontuação exceda o limite vão ser registadas e, caso o
\emph{reporting level} seja suficiente, reportadas.
\end{description}

\confoption{Outros ficheiros}~\\

De seguida são apresentados outros ficheiros de configuração do
dansguardian com as respectivas finalidades:

\begin{description}
\item[exceptionsitelist]~\\
Este ficheiro contém uma lista de domínios para os quais o dansguardian
não vai fazer filtragem de conteúdo.

De notar que não se deve colocar o http:// ou o www início de cada
entrada.

\item[exceptioniplist]~\\
Contém uma lista de ips de origem que não serão filtrados, por exemplo,
o ip de um administrador.

De notar que este ficheiro só tem relevância caso o dansguardian receba
directamente os pedidos dos browsers ou caso o proxy anterior envie o
\emph{X-ForwardFor}.

\item[exceptionuserlist]~\\
Utilizadores que não vão ser filtrados (só funciona com autenticação
básica ou \emph{ident}).

\item[exceptionphraselist]~\\
Se alguma destas palavras aparecer no conteúdo de uma página, a
filtragem vai ser desactivada para este conteúdo.

Como tal é preciso algum cuidado ao adicionar entradas novas neste
ficheiro. 
Uma melhor solução poderia ser colocar um valor negativo no
\emph{weightedphraselist}.

\item[exceptionurllist]~\\
\emph{URL's} colocados nesta página não vão ser alvo de filtragem.

\item[bannediplist]~\\
Endereços IP de clientes a negar o acesso Web.

Só devem ser aqui colocados ips e não \emph{hostnames}.

\item[bannedphraselist]~\\
Os conteúdos a negar na filtragem podem ser modificados no ficheiro
\emph{bannedphraselist}.

Este ficheiro já inclui alguns tipos de palavras como se pode ver pelo
extracto:

\begin{Output}
.Include</etc/dansguardian/phraselists/pornography/banned>
.Include</etc/dansguardian/phraselists/illegaldrugs/banned>
.Include</etc/dansguardian/phraselists/gambling/banned>
\end{Output}

\item[banneduserlist]~\\
Nomes de utilizadores que, caso a autenticação básica esteja activada,
lhes será negado o acesso ao exterior.
Funcionalidades relativas a ips de origem e utilizadores apenas fazem
sentido quando o dansguardian é o proxy de entrada em vez do squid.

\item[bannedmimetypelist]~\\
Contém uma lista de \emph{MIME-types} banidos.

Se um pedido a um \emph{URL} retornar um \emph{MIME-type} que esteja
nesta lista vai ser bloqueado pelo DansGuardian.

Permite, por exemplo, bloquear filmes, ficheiros de música, etc.
É aconselhável não banir os \emph{MIME-types} text/html ou image/*.

\item[bannedextensionlist]~\\
Contém uma lista de extensões de ficheiros banidos.

Se um \emph{URL} termina com uma extensão desta lista vai ser
bloqueado pelo DansGuardian.

\item[bannedregexpurllist]~\\
Permite banir determinados endereços baseado em expressões regulares.
\end{description}

\subsubsection{Operações sobre o serviço}

\ServiceCommands{dansguardian}

Para que o serviço se inicie correctamente é necessário que o proxy
de saída esteja a funcionar.

Todas estas alterações podem ser realizadas no frontend da \textsf{ETFW}.
