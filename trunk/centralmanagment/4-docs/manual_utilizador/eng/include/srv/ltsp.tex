\subsection{Serviço LTSP}

O \emph{LTSP} é um serviço de implementação de clientes \emph{desktop} minimalista para ambientes Linux.


Os ficheiros de configuração estão localizados em:

\begin{Verbatim}[commandchars=\\\{\}]
/etc/ltsp/
\end{Verbatim}

O \emph{root} para os \emph{thin-clients} está localizado em:

\begin{Verbatim}[commandchars=\\\{\}]
/opt/ltsp/
\end{Verbatim}

Os clientes ligam-se por \emph{PXE}, por isso é necessário o \emph{DHCP} estar activo e fornecer acesso \emph{PXE} por \emph{TFTP} em que as configurações podem ser feitas em:

\begin{Verbatim}[commandchars=\\\{\}]
/var/lib/tftpboot/ltsp/
\end{Verbatim}

Para activar os serviços deve-se correr os seguintes commandos:
\begin{Verbatim}[commandchars=\\\{\}]

for service in xinetd ltsp-dhcpd rpcbind nfs sshd; \
do
    service $service start;
done

for server in ldminfod nbdrootd nbdswapd tftp;
do
    service $server start;
done

\end{Verbatim}

Para actualizar a \emph{root} dos \emph{thin-clients} deve-se correr os seguintes commandos:

\begin{Verbatim}[commandchars=\\\{\}]

setarch i386 chroot /opt/ltsp/i386

yum update
ltsp-rewrap-latest-kernel

exit

ltsp-update-kernels

\end{Verbatim}

A instalação de uma nova \emph{root} de \emph{thin-clients} pode ser feita da seguinte forma:

\begin{Verbatim}[commandchars=\\\{\}]
ltsp-build-client --kickstart=/etc/ltsp/kickstart/ltsp-i386.ks  --chroot=nova-i386
\end{Verbatim}

Informações adicionais podem ser obtidas em:\\ \begin{normalsize}\sffamily\href{http://www.ltsp.org}{www.ltsp.org}\end{normalsize}

