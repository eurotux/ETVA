\subsection{IPSEC}

O \emph{ipsec}\footnote{(IP Security Protocol} é uma extensão do protocolo IP que tem como objectivo o fornecimento de privacidade do utilizador (aumentando a confiabilidade das informações fornecidas pelo mesmo através da Internet), integridade dos dados (garantindo que o mesmo conteúdo que chegou ao destino seja o mesmo da origem) e autenticidade das informações ou \emph{identity spoofing} (garantia de que uma pessoa é quem diz ser), quando se transferem informações através de redes IP pela Internet.

O ficheiro de configuração deste serviço é:

\begin{Verbatim}[commandchars=\\\{\}]
/etc/racoon/racoon.conf
\end{Verbatim}

O comando para controlar este serviço (iniciar/parar) é o seguinte:

\begin{Verbatim}[commandchars=\\\{\}]
# /etc/init.d/racoon start
# /etc/init.d/racoon stop
\end{Verbatim}

